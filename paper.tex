% \documentclass[11pt]{article}
\documentclass{article}

% if you need to pass options to natbib, use, e.g.:
%     \PassOptionsToPackage{numbers, compress}{natbib}
% before loading neurips_2020

 
% ready for submission
% \usepackage{neurips_2020}

% to compile a preprint version, e.g., for submission to arXiv, add add the
% [preprint] option:
%     \usepackage[preprint]{neurips_2020}

% to compile a camera-ready version, add the [final] option, e.g.:
%     \usepackage[final]{neurips_2020}

% to avoid loading the natbib package, add option nonatbib:
%     \usepackage[nonatbib]{neurips_2020}

\usepackage[utf8]{inputenc} % allow utf-8 input
\usepackage[T1]{fontenc}    % use 8-bit T1 fonts
%\usepackage{hyperref}       % hyperlinks 
\usepackage{url}            % simple URL typesetting
\usepackage{booktabs}       % professional-quality tables
\usepackage{amsfonts}       % blackboard math symbols
\usepackage{nicefrac}       % compact symbols for 1/2, etc.
\usepackage{microtype}      % microtypography

\usepackage{tocloft}            % TOC spacing

%\usepackage{txfont}

\usepackage{enumitem}

\usepackage{breakcites}

\usepackage{etoolbox}
\usepackage{comment}
\newtoggle{draft}
\togglefalse{draft}
\newcommand{\draft}[1]{\iftoggle{draft}{#1}{}}


% \mathscr
\usepackage{mathrsfs}

\usepackage{algorithm}
\usepackage{verbatim}
\usepackage[noend]{algpseudocode}
\newcommand{\multiline}[1]{\parbox[t]{\dimexpr\linewidth-\algorithmicindent}{#1}}

\usepackage{multicol}

\usepackage{colortbl}

\usepackage{setspace}

\usepackage{transparent}

\usepackage{inconsolata}
\usepackage[scaled=.90]{helvet}
%\usepackage{fontspec}
% \usepackage{helvet}
\usepackage{xspace}

\usepackage{pifont}
\newcommand{\cmark}{\ding{51}}%
\newcommand{\xmark}{\ding{55}}%
\usepackage{bm}
\newcommand{\x}{\bm{x}}

% \usepackage{eufrak}
% \usepackage{fontspec,unicode-math}
% \setmathfont[range=\mathfrak]{Old English Text MT}

%%%%%%%%%%%%%%%%%%%%%%%%%%%%%%%%%%%%%%%%%%%%%%

% Better looking lowercase mathfrak
% https://tex.stackexchange.com/questions/398710/alternative-separate-font-to-mathfrak-for-small-letters

\DeclareFontFamily{U}{jkpmia}{}
\DeclareFontShape{U}{jkpmia}{m}{it}{<->s*jkpmia}{}
\DeclareFontShape{U}{jkpmia}{bx}{it}{<->s*jkpbmia}{}
\DeclareMathAlphabet{\mathfrak}{U}{jkpmia}{m}{it}
\SetMathAlphabet{\mathfrak}{bold}{U}{jkpmia}{bx}{it}

%%%%%%%%%%%%%%%%%%%%%%%%%%%%%%%%%%%%%%%%%%%%%%

%\usepackage{geometry}
\usepackage[letterpaper, left=1in, right=1in, top=1in, bottom=1in]{geometry}

\PassOptionsToPackage{hypertexnames=false}{hyperref}  % compatible with cref for multiple algo
\usepackage{parskip}

\usepackage[dvipsnames]{xcolor}
\usepackage[colorlinks=true, linkcolor=blue!70!black, citecolor=blue!70!black,urlcolor=black,breaklinks=true]{hyperref}
\usepackage{microtype}
\usepackage{hhline}

\makeatletter
\newcommand{\neutralize}[1]{\expandafter\let\csname c@#1\endcsname\count@}
\makeatother

   \newenvironment{lemmod}[2]
  {\renewcommand{\thelemma}{\ref*{#1}#2}%
   \neutralize{lemma}\phantomsection
   \begin{lemma}}
  {\end{lemma}}


\newenvironment{thmmod}[2]
  {\renewcommand{\thetheorem}{\ref*{#1}#2}%
   \neutralize{theorem}\phantomsection
   \begin{theorem}}
  {\end{theorem}}


\newenvironment{bodythmmod}[2]
  {\renewcommand{\thebodytheorem}{\ref*{#1}#2}%
   \neutralize{bodytheorem}\phantomsection
   \begin{theorem}}
  {\end{theorem}}

    \newenvironment{propmod}[2]
  {\renewcommand{\theproposition}{\ref*{#1}#2}%
   \neutralize{proposition}\phantomsection
   \begin{proposition}}
  {\end{proposition}}

     \newenvironment{defnmod}[2]
  {\renewcommand{\thedefinition}{\ref*{#1}#2}%
   \neutralize{definition}\phantomsection
   \begin{definition}}
  {\end{definition}}

   \newenvironment{asmmod}[2]
  {\renewcommand{\theassumption}{\ref*{#1}#2}%
   \neutralize{assumption}\phantomsection
   \begin{assumption}}
  {\end{assumption}}

  \newenvironment{cormod}[2]
  {\renewcommand{\thecorollary}{\ref*{#1}#2}%
   \neutralize{corollary}\phantomsection
   \begin{corollary}}
  {\end{corollary}}
  
   \newenvironment{condmod}[2]
  {\renewcommand{\thecondition}{\ref*{#1}#2}%
   \neutralize{condition}\phantomsection
   \begin{condition}}
  {\end{condition}}


% algorithms
\usepackage{algorithm}
% \usepackage{algpseudocode}

%  \let\oldparagraph\paragraph
%  \renewcommand{\paragraph}[1]{\oldparagraph{#1.}}

\usepackage{natbib}
\bibliographystyle{plainnat}
\bibpunct{(}{)}{;}{a}{,}{,}

\usepackage{amsthm}
\usepackage{mathtools}
\usepackage{amsmath}
\usepackage{bbm}
\usepackage{amsfonts}
\usepackage{amssymb}
% \let\vec\undefined
% \usepackage{MnSymbol} %Actually conflicts with amssymb and others

\usepackage{xpatch}

%%% theorems

\usepackage{thmtools}
\usepackage{thm-restate}
\declaretheorem[name=Theorem,parent=section]{theorem}
\declaretheorem[name=Lemma,parent=section]{lemma}
\declaretheorem[name=Assumption, parent=section]{assumption}
\declaretheorem[name=Condition, parent=section]{condition}
% \declaretheorem[name=Example,style=definition]{example}
% \declaretheorem[qed=$\triangleleft$,name=Example,style=definition, thmbox=S]{example}
\declaretheorem[qed=$\triangleleft$,name=Example,style=definition, parent=section]{example}
\declaretheorem[name=Remark,style=definition, parent=section]{remark}
\declaretheorem[name=Proposition, parent=section]{proposition}
\declaretheorem[name=Fact, parent=section]{fact}

% This needs to come after we include thmtools
\makeatletter
  \renewenvironment{proof}[1][Proof]%
  {%
   \par\noindent{\bfseries\upshape {#1.}\ }%
  }%
  {\qed\newline}
  \makeatother

\theoremstyle{definition}  %Sets style of subsequent newtheorems to 'definition'
\newtheorem{exercise}{Exercise}
\newtheorem{claim}{Claim}
% \newtheorem{lemma}{Lemma}[section]
%\newtheorem{assumption}{Assumption}
\newtheorem{conjecture}{Conjecture}
\newtheorem{corollary}{Corollary}[section]
% \newtheorem{proposition}{Proposition}[section]
% \newtheorem{fact}{Fact}
%\newtheorem{model}{Model}
%\newtheorem{problem}{Problem}
% \newtheorem{assumption}{Assumption}
\newtheorem{problem}{Problem}
\newtheorem{question}{Question}
\newtheorem{model}{Model}

% \newtheorem{examplex}{Example}
% \newenvironment{example}
%   {\pushQED{\qed}\renewcommand{\qedsymbol}{$\triangle$}\examplex}
%   {\popQED\endexamplex}

\theoremstyle{plain}
% \newtheorem{remark}{Remark}
% \newtheorem{example}{Example}
% \newtheorem{theorem}{Theorem}[section]
\newtheorem{definition}{Definition}[section]
\newtheorem{openproblem}{Open Problem}

\xpatchcmd{\proof}{\itshape}{\normalfont\proofnameformat}{}{}
\newcommand{\proofnameformat}{\bfseries}

\newcommand{\notimplies}{\nRightarrow}

%%% prettyref

\usepackage[nameinlink,capitalize]{cleveref}

\newcommand{\pref}[1]{\cref{#1}}
\newcommand{\pfref}[1]{Proof of \pref{#1}}
\newcommand{\savehyperref}[2]{\texorpdfstring{\hyperref[#1]{#2}}{#2}}

\crefformat{equation}{#2(#1)#3}
\Crefformat{equation}{#2(#1)#3}

\crefname{fact}{Fact}{Facts}

\Crefformat{figure}{#2Figure #1#3}
\Crefformat{assumption}{#2Assumption #1#3}

% Fix hyperref in section titles.
\usepackage{crossreftools}
\pdfstringdefDisableCommands{%
    \let\Cref\crtCref
    \let\cref\crtcref
}
\newcommand{\preft}[1]{\crtcref{#1}}


% \usepackage{prettyref}
% \newcommand{\pref}[1]{\prettyref{#1}}
% \newcommand{\pfref}[1]{Proof of \prettyref{#1}}
% \newcommand{\savehyperref}[2]{\texorpdfstring{\hyperref[#1]{#2}}{#2}}
% \newrefformat{eq}{\savehyperref{#1}{\textup{(\ref*{#1})}}}
% \newrefformat{eqn}{\savehyperref{#1}{Equation~\ref*{#1}}}
% \newrefformat{lem}{\savehyperref{#1}{Lemma~\ref*{#1}}}p
% \newrefformat{def}{\savehyperref{#1}{Definition~\ref*{#1}}}
% \newrefformat{line}{\savehyperref{#1}{line~\ref*{#1}}}
% \newrefformat{thm}{\savehyperref{#1}{Theorem~\ref*{#1}}}
% \newrefformat{corr}{\savehyperref{#1}{Corollary~\ref*{#1}}}
% \newrefformat{cor}{\savehyperref{#1}{Corollary~\ref*{#1}}}
% \newrefformat{sec}{\savehyperref{#1}{Section~\ref*{#1}}}
% \newrefformat{app}{\savehyperref{#1}{Appendix~\ref*{#1}}}
% \newrefformat{ass}{\savehyperref{#1}{Assumption~\ref*{#1}}}
% \newrefformat{ex}{\savehyperref{#1}{Example~\ref*{#1}}}
% \newrefformat{fig}{\savehyperref{#1}{Figure~\ref*{#1}}}
% \newrefformat{alg}{\savehyperref{#1}{Algorithm~\ref*{#1}}}
% \newrefformat{rem}{\savehyperref{#1}{Remark~\ref*{#1}}}
% \newrefformat{conj}{\savehyperref{#1}{Conjecture~\ref*{#1}}}
% \newrefformat{prop}{\savehyperref{#1}{Proposition~\ref*{#1}}}
% \newrefformat{proto}{\savehyperref{#1}{Protocol~\ref*{#1}}}
% \newrefformat{prob}{\savehyperref{#1}{Problem~\ref*{#1}}}
% \newrefformat{claim}{\savehyperref{#1}{Claim~\ref*{#1}}}

%  Pro version of declarepaireddelimiter
%  https://tex.stackexchange.com/questions/136749/super-and-subscripts-with-declarepaireddelimiter/136767#136767
\usepackage{xparse}

\ExplSyntaxOn
\DeclareDocumentCommand{\XDeclarePairedDelimiter}{mm}
 {
  \__egreg_delimiter_clear_keys: % reset to the default
  \keys_set:nn { egreg/delimiters } { #2 }
  \use:x % we want to expand the values of the token variables set with the keys
   {
    \exp_not:n {\NewDocumentCommand{#1}{sO{}m} }
     {
      \exp_not:n { \IfBooleanTF{##1} }
       {
        \exp_not:N \egreg_paired_delimiter_expand:nnnn
         { \exp_not:V \l_egreg_delimiter_left_tl }
         { \exp_not:V \l_egreg_delimiter_right_tl }
         { \exp_not:n { ##3 } }
         { \exp_not:V \l_egreg_delimiter_subscript_tl }
       }
       {
        \exp_not:N \egreg_paired_delimiter_fixed:nnnnn 
         { \exp_not:n { ##2 } }
         { \exp_not:V \l_egreg_delimiter_left_tl }
         { \exp_not:V \l_egreg_delimiter_right_tl }
         { \exp_not:n { ##3 } }
         { \exp_not:V \l_egreg_delimiter_subscript_tl }
       }
     }
   }
 }

\keys_define:nn { egreg/delimiters }
 {
  left      .tl_set:N = \l_egreg_delimiter_left_tl,
  right     .tl_set:N = \l_egreg_delimiter_right_tl,
  subscript .tl_set:N = \l_egreg_delimiter_subscript_tl,
 }

\cs_new_protected:Npn \__egreg_delimiter_clear_keys:
 {
  \keys_set:nn { egreg/delimiters } { left=.,right=.,subscript={} }
 }

\cs_new_protected:Npn \egreg_paired_delimiter_expand:nnnn #1 #2 #3 #4
 {% Fix the spacing issue with \left and \right (D. Arsenau, P. Stephani and H. Oberdiek)
  \mathopen{}
  \mathclose\c_group_begin_token
   \left#1
   #3
   \group_insert_after:N \c_group_end_token
   \right#2
   \tl_if_empty:nF {#4} { \c_math_subscript_token {#4} }
 }
\cs_new_protected:Npn \egreg_paired_delimiter_fixed:nnnnn #1 #2 #3 #4 #5
 {
  \mathopen{#1#2}#4\mathclose{#1#3}
  \tl_if_empty:nF {#5} { \c_math_subscript_token {#5} }
 }
\ExplSyntaxOff

% Example
\XDeclarePairedDelimiter{\supnorm}{
  left=\lVert,
  right=\rVert,
  subscript=\infty
  }
%%% Local Variables:
%%% mode: latex
%%% TeX-master: "paper"
%%% End:

% !TEX root = paper.tex

% Math delimiters
\DeclarePairedDelimiter{\abs}{\lvert}{\rvert} %
\DeclarePairedDelimiter{\brk}{[}{]}
\DeclarePairedDelimiter{\crl}{\{}{\}}
\DeclarePairedDelimiter{\prn}{(}{)}
\DeclarePairedDelimiter{\nrm}{\|}{\|}
\DeclarePairedDelimiter{\tri}{\langle}{\rangle}
\DeclarePairedDelimiter{\dtri}{\llangle}{\rrangle}

\DeclarePairedDelimiter{\ceil}{\lceil}{\rceil}
\DeclarePairedDelimiter{\floor}{\lfloor}{\rfloor}

\DeclarePairedDelimiter{\nrmlip}{\|}{\|_{\mathrm{Lip}}}

% \DeclareMathOperator{\E}{\mathbb{E}} %expecation
\let\Pr\undefined
\let\P\undefined
\DeclareMathOperator{\En}{\mathbb{E}}
\DeclareMathOperator{\Enbar}{\widebar{\mathbb{E}}}
%\DeclareMathOperator{\Ep}{{\widetilde{\mathbb{E}}}}
\newcommand{\Ep}{\wt{\bbE}}
% \DeclareMathOperator*{\Eh}{\widehat{\mathbb{E}}}
\newcommand{\Eh}{\wh{\bbE}}
\DeclareMathOperator{\P}{P}
\DeclareMathOperator{\Pr}{Pr}

\DeclareMathOperator*{\maximize}{maximize} % * Places subscript
% directly under operator
\DeclareMathOperator*{\minimize}{minimize} % * Places subscript directly under operator

% Arg<x>
% \DeclareMathOperator*{\liminf}{lim\,inf} % 
% \DeclareMathOperator*{\limsup}{lim\,sup}             

\DeclareMathOperator*{\argmin}{arg\,min} % * Places subscript directly under operator
\DeclareMathOperator*{\argmax}{arg\,max}             

% styles
\newcommand{\mc}[1]{\mathcal{#1}}
\newcommand{\mb}[1]{\boldsymbol{#1}}
\newcommand{\wt}[1]{\widetilde{#1}}
\newcommand{\wh}[1]{\widehat{#1}}
\newcommand{\wb}[1]{\widebar{#1}}


% Special letters: blackboard, mathcal, widehat % djhsu magic
\def\ddefloop#1{\ifx\ddefloop#1\else\ddef{#1}\expandafter\ddefloop\fi}
\def\ddef#1{\expandafter\def\csname bb#1\endcsname{\ensuremath{\mathbb{#1}}}}
\ddefloop ABCDEFGHIJKLMNOPQRSTUVWXYZ\ddefloop
\def\ddefloop#1{\ifx\ddefloop#1\else\ddef{#1}\expandafter\ddefloop\fi}
\def\ddef#1{\expandafter\def\csname b#1\endcsname{\ensuremath{\mathbf{#1}}}}
\ddefloop ABCDEFGHIJKLMNOPQRSTUVWXYZ\ddefloop
\def\ddef#1{\expandafter\def\csname sf#1\endcsname{\ensuremath{\mathsf{#1}}}}
\ddefloop ABCDEFGHIJKLMNOPQRSTUVWXYZ\ddefloop
\def\ddef#1{\expandafter\def\csname c#1\endcsname{\ensuremath{\mathcal{#1}}}}
\ddefloop ABCDEFGHIJKLMNOPQRSTUVWXYZ\ddefloop
\def\ddef#1{\expandafter\def\csname h#1\endcsname{\ensuremath{\widehat{#1}}}}
\ddefloop ABCDEFGHIJKLMNOPQRSTUVWXYZ\ddefloop
\def\ddef#1{\expandafter\def\csname hc#1\endcsname{\ensuremath{\widehat{\mathcal{#1}}}}}
\ddefloop ABCDEFGHIJKLMNOPQRSTUVWXYZ\ddefloop
\def\ddef#1{\expandafter\def\csname t#1\endcsname{\ensuremath{\widetilde{#1}}}}
\ddefloop ABCDEFGHIJKLMNOPQRSTUVWXYZ\ddefloop
\def\ddef#1{\expandafter\def\csname tc#1\endcsname{\ensuremath{\widetilde{\mathcal{#1}}}}}
\ddefloop ABCDEFGHIJKLMNOPQRSTUVWXYZ\ddefloop
% scr
\def\ddefloop#1{\ifx\ddefloop#1\else\ddef{#1}\expandafter\ddefloop\fi}
\def\ddef#1{\expandafter\def\csname scr#1\endcsname{\ensuremath{\mathscr{#1}}}}
\ddefloop ABCDEFGHIJKLMNOPQRSTUVWXYZ\ddefloop


% one-off macros
\newcommand{\ls}{\ell}
\newcommand{\ind}{\mathbbm{1}}    %Indicator
\newcommand{\pmo}{\crl*{\pm{}1}}
\newcommand{\eps}{\epsilon}
\newcommand{\veps}{\varepsilon}
\newcommand{\vphi}{\varphi}

\newcommand{\ldef}{\vcentcolon=}
\newcommand{\rdef}{=\vcentcolon}


%%%%%%%%%%%%%%%%%%%%%%%%%%%%%%%%%%%%%%%%%%%%%%%%%
% Turn asterisk into \star %%%%%%%%%%%%%%%%%%%%%%%%%%%%%%%
%%%%%%%%%%%%%%%%%%%%%%%%%%%%%%%%%%%%%%%%%%%%%%%%%
\begin{comment}
% See https://tex.stackexchange.com/questions/414919/redefining-asterisk-in-math-mode
\usepackage{xpatch}

\makeatletter
\newcommand{\DeclareMathActive}[2]{%
  % #1 is the character, #2 is the definition
  \expandafter\edef\csname keep@#1@code\endcsname{\mathchar\the\mathcode`#1 }
  \begingroup\lccode`~=`#1\relax
  \lowercase{\endgroup\def~}{#2}%
  \AtBeginDocument{\mathcode`#1="8000 }%
}
\newcommand{\std}[1]{\csname keep@#1@code\endcsname}
\patchcmd{\newmcodes@}{\mathcode`\-\relax}{\std@minuscode\relax}{}{\ddt}
\AtBeginDocument{\edef\std@minuscode{\the\mathcode`-}}
\makeatother
\DeclareMathActive{*}{\ast}
\renewcommand{\ast}{\star}
\end{comment}

%%% Local Variables:
%%% mode: latex
%%% TeX-master: "paper"
%%% End:

% !TEX root = paper.tex

%%%%%%%%%%%%%%%%%%%%%%%%%%%%%%%%%%%%%%%%%%%%%%%%%%%%
% Paper writing %%%%%%%%%%%%%%%%%%%%%%%%%%%%%%%%%%%%%%%%%
%%%%%%%%%%%%%%%%%%%%%%%%%%%%%%%%%%%%%%%%%%%%%%%%%%%%


\newcommand{\Drho}[2]{\rho\prn*{#1\dmid{}#2}}

\newcommand{\pibar}{\wb{\pi}}

\newcommand{\Cpush}{C_{\mathsf{push}}}

  \newcommand{\sfrak}{\mathfrak{s}}
  \newcommand{\afrak}{\mathfrak{a}}
  \newcommand{\bfrak}{\mathfrak{b}}
  \newcommand{\cfrak}{\mathfrak{c}}
  \newcommand{\xfrak}{\mathfrak{x}}
  \newcommand{\yfrak}{\mathfrak{y}}
  \newcommand{\zfrak}{\mathfrak{z}}
    \newcommand{\bxi}{\mb{\xi}}
  \newcommand{\bs}{\mathbf{s}}
  \newcommand{\ba}{\mathbf{a}}
  \newcommand{\bx}{\mathbf{x}}
  \newcommand{\br}{\mathbf{r}}


\newcommand{\vepsstat}{\veps_{\mathrm{stat}}}

\newcommand{\ppr}{p_{\mathrm{pr}}}
\newcommand{\ppo}{p_{\mathrm{po}}}

\newcommand{\gammastar}{\gamma^{\star}}

\newcommand{\compc}[1][\veps]{\mathsf{dec}_{#1}}

\newcommand{\var}{\mathrm{Var}}
\newcommand{\Var}{\var}
\newcommand{\varplus}{\mathrm{Var}_{\sss{+}}}
\newcommand{\Varplus}{\varplus}

\newcommand{\Varm}[1][M]{\Var\sups{#1}}
\newcommand{\Varmbar}[1][\Mbar]{\Var\sups{#1}}

%%%%%%%%%%%%%%%%%%%%%%%%%%%%%%%%%%%%%%%%%%%%%%%%%%%%
% RL section %%%%%%%%%%%%%%%%%%%%%%%%%%%%%%%%%%%%%%%%%%%
%%%%%%%%%%%%%%%%%%%%%%%%%%%%%%%%%%%%%%%%%%%%%%%%%%%%

% Lower bound
\newcommand{\pmbar}{p_{\sMbar}}
\newcommand{\pmdist}{p\subs{M}}
% \renewcommand{\pm}{\phat^{\sM}}
\newcommand{\PM}{\bbP\sups{M}}
\newcommand{\PMbar}{\bbP\sups{\Mbar}}

% \newcommand{\delm}{\Delta_{M}}
% \newcommand{\delmbar}{\Delta_{\Mbar}}
% \newcommand{\gammat}{\frac{\gamma}{T}}

% Refined results
\newcommand{\Dbi}[2]{D_{\mathrm{bi}}\prn*{#1\dmid#2}}
\newcommand{\Dbishort}{D_{\mathrm{bi}}}
\newcommand{\cQm}{\cQ_{\cM}}
\newcommand{\cQmh}[1][h]{\cQ_{\cM;#1}}
\newcommand{\cTm}[1][M]{\cT\sups{#1}}
\newcommand{\cTmstar}[1][\Mstar]{\cT\sups{#1}}

\newcommand{\Confone}{R_1}
\newcommand{\Conftwo}{R_2}
\newcommand{\compbi}{\compgen[\mathrm{bi}]}
\newcommand{\compbidual}{\compgendual[\mathrm{bi}]}

\newcommand{\bloss}{\cL}

\newcommand{\cGmbar}{\cG\sups{\Mbar}}
\newcommand{\gmmbar}[1][M]{g\sups{#1;\Mbar}}
\newcommand{\Lbr}{L_{\mathrm{br}}}
\newcommand{\Lbrfull}{L_{\mathrm{br}}(\cM)}
\newcommand{\Lbrfulls}{L^{2}_{\mathrm{br}}(\cM)}

\newcommand{\mum}[1][M]{\mu\sups{#1}}
\newcommand{\wm}[1][M]{w\sups{#1}}
\newcommand{\phim}[1][M]{\phi\sups{#1}}

\newcommand{\AlgEsth}[1][h]{\mathrm{\mathbf{Alg}}_{\mathsf{Est};#1}}
\newcommand{\EstHelh}[1][h]{\mathrm{\mathbf{Est}}_{\mathsf{H};#1}}

\newcommand{\thetam}[1][M]{\theta(#1)}
\newcommand{\thetambar}[1][\Mbar]{\theta(#1)}

\newcommand{\Copt}{C_{\mathrm{opt}}}

\newcommand{\Calpha}{C_{\alpha}}

\newcommand{\dimbi}{d_{\mathrm{bi}}}
\newcommand{\Lbi}{L_{\mathrm{bi}}}
\newcommand{\dimbifull}{d_{\mathrm{bi}}(\cM)}
\newcommand{\dimbifulls}[1][2]{d^{#1}_{\mathrm{bi}}(\cM)}
\newcommand{\Lbifull}{L_{\mathrm{bi}}(\cM)}
\newcommand{\Lbifulls}{L^{2}_{\mathrm{bi}}(\cM)}

\newcommand{\piest}{\pi^{\mathrm{est}}}
\newcommand{\piestm}[1][M]{\pi^{\mathrm{est}}_{\sss{#1}}}
\newcommand{\lest}{\ell^{\mathrm{est}}}
\newcommand{\lestm}[1][M]{\ell^{\mathrm{est}}_{\sss{#1}}}

\newcommand{\Piest}{\Pi^{\mathrm{est}}}

\newcommand{\qbar}{\bar{q}}
\newcommand{\qopt}{q^{\mathrm{opt}}}

\newcommand{\pialpha}{\pi^{\alpha}}
\newcommand{\pialpham}[1][M]{\pi^{\alpha}_{\sss{#1}}}


%%%%%%%%%%%%%%%%%%%%%%%%%%%%%%%%%%%%%%%%%%%%%%%%%%%%
% Contextual section %%%%%%%%%%%%%%%%%%%%%%%%%%%%%%%%%%%%%
%%%%%%%%%%%%%%%%%%%%%%%%%%%%%%%%%%%%%%%%%%%%%%%%%%%%

\newcommand{\con}{x}
\newcommand{\Cspace}{\cX}

% \newcommand{\cMx}[1][x]{\cM_{#1}}
% \newcommand{\cMhatx}[1][x]{\cMhat_{#1}}
\newcommand{\cMx}[1][x]{\cM|_{#1}}
\newcommand{\cMhatx}[1][x]{\cMhat|_{#1}}
\newcommand{\cFmx}[1][x]{\cFm|_{#1}}

% \newcommand{\cpol}{\rho}
\newcommand{\cpol}{\mb{\pi}}
\newcommand{\cpolstar}{\cpol^{\star}}
\newcommand{\cpolm}[1][M]{\cpol\subs{#1}}

\newcommand{\RegCDM}{\Reg_{\mathsf{DM}}}
\newcommand{\EstCD}{\mathrm{\mathbf{Est}}_{\mathsf{D}}}

%%%%%%%%%%%%%%%%%%%%%%%%%%%%%%%%%%%%%%%%%%%%%%%%%%%%
% Bandit section %%%%%%%%%%%%%%%%%%%%%%%%%%%%%%%%%%%%%%%%
%%%%%%%%%%%%%%%%%%%%%%%%%%%%%%%%%%%%%%%%%%%%%%%%%%%%

% \newcommand{\relu}{\textsf{relu}}
\newcommand{\relu}{\mathsf{relu}}

\newcommand{\sdis}{\mb{\theta}}
\newcommand{\rdis}{\mb{\psi}}

\newcommand{\Deltanot}{\Delta_0}

\newcommand{\Delm}{\Delta\subs{M}}

% \newcommand{\Star}{\mathsf{Sdim}}
\newcommand{\Star}{\mathfrak{s}}
\newcommand{\StarCheck}{\check{\Star}}
\newcommand{\StarDim}{\Star}
\newcommand{\Stardim}{\Star}
\newcommand{\El}{\mathfrak{e}}
\newcommand{\ElCheck}{\check{\El}}
\newcommand{\ElDim}{\El}
\newcommand{\Eldim}{\El}

% Illustrative examples section

\newcommand{\errm}[1][M]{\mathrm{err}\sups{#1}}

\newcommand{\hardfamily}{$(\alpha,\beta,\delta)$-family\xspace}
\newcommand{\hardu}{u}
\newcommand{\hardv}{v}

\newcommand{\igwshort}{IGW\xspace}
\newcommand{\igwtext}{inverse gap weighting\xspace}
\newcommand{\pcigw}{\textsf{PC-IGW}\xspace}
\newcommand{\pcigwb}{\textsf{PC-IGW.Bilinear}\xspace}
\newcommand{\pcigwtext}{policy cover inverse gap weighting\xspace}

\newcommand{\igwspanner}{\textsf{IGW-Spanner}\xspace}
\newcommand{\igwargmax}{\textsf{IGW-ArgMax}\xspace}

% \newcommand{\oracle}{\mathsf{Oracle}\xspace}
\newcommand{\oracle}{\mathrm{\mathbf{Alg}}_{\mathsf{Plan}}\xspace}

\newcommand{\I}{\mathrm{Err}_{\mathrm{I}}}
\newcommand{\II}{\mathrm{Err}_{\mathrm{II}}}
\newcommand{\III}{\mathrm{Err}_{\mathrm{III}}}

% Minimax regret

\newcommand{\MinimaxReg}{\mathfrak{M}(\cM,T)}
% \newcommand{\MinimaxRegBayes}{\mathfrak{M}^{\mathrm{Bayes}}(\cM,T)}
\newcommand{\MinimaxRegBayes}{\underbar{\mathfrak{M}}(\cM,T)}


\renewcommand{\emptyset}{\varnothing}

\newcommand{\NullObs}{\crl{\emptyset}}

% Probability
\newcommand{\filt}{\mathscr{F}}
\newcommand{\ffilt}{\mathscr{F}}
\newcommand{\gfilt}{\mathscr{G}}
\newcommand{\hfilt}{\mathscr{H}}

\newcommand{\hist}{\mathcal{H}}

\newcommand{\Asig}{\mathscr{P}}
% \newcommand{\Rsig}{\mathfrak{B}(\bbR)}
\newcommand{\Rsig}{\mathscr{R}}
\newcommand{\Osig}{\mathscr{O}}

\newcommand{\borel}{\mathfrak{B}}

\newcommand{\Hspace}{\Omega}
\newcommand{\Hsig}{\filt}

\newcommand{\dom}{\nu}

\newcommand{\dens}{m}
\newcommand{\densm}[1][M]{m^{\sss{#1}}}

\newcommand{\abscont}{V(\cM)}
\newcommand{\abscontp}{V(\cM')}


% Lower bound proof
% \newcommand{\Ct}{C(T)}

\newcommand{\Vmbar}{V^{\sMbar}}

% General framework

\newcommand{\vepsup}{\bar{\veps}}
% \newcommand{\ubar}[1]{\underaccent{\bar}{#1}}
\newcommand{\vepslow}{\ubar{\veps}}

\newcommand{\vepsg}{\veps_{\gamma}}

\newcommand{\vepsupg}{\bar{\veps}_{\gamma}}
\newcommand{\vepslowg}{\ubar{\veps}_{\gamma}}
% \newcommand{\vepslow}{\veps_1}

\newcommand{\gameval}[1]{\cV_{\gamma}\sups{#1}}

\newcommand{\Framework}{Decision Making with Structured Observations\xspace}
\newcommand{\FrameworkShort}{DMSO\xspace}

\newcommand{\learner}{learner\xspace}

% \newcommand{\Cloc}{C_{\mathrm{loc}}}
% \newcommand{\Cloc}{c_{\mathrm{loc}}}
\newcommand{\Cloc}{c_{\ell}}
\newcommand{\Cglobal}{C(\cM)}

\newcommand{\act}{\pi}
\newcommand{\Act}{\Pi}
\newcommand{\ActSpace}{\Pi}
\newcommand{\ActSize}{\abs{\Pi}}
\newcommand{\ActSizeB}{\abs{\cA}}
\newcommand{\ActSizeRL}{\abs{\cA}}
\newcommand{\ActDist}{\Delta(\Act)}
\newcommand{\action}{decision\xspace}
\newcommand{\actions}{decisions\xspace}
\newcommand{\actstar}{\pistar}

% \newcommand{\Aspace}{\cA}

% Actions for bandit setting
\newcommand{\actb}{a}

\newcommand{\StateSize}{\abs{\cS}}

\newcommand{\obs}{o}
\newcommand{\Obs}{\mathcal{\cO}}
\newcommand{\ObsSpace}{\mathcal{\cO}}
\newcommand{\Ospace}{\mathcal{\cO}}

% \newcommand{\RewardSpace}{\bbR}
\newcommand{\RewardSpace}{\cR}
\newcommand{\RSpace}{\RewardSpace}
\newcommand{\Rspace}{\RewardSpace}

\newcommand{\Model}{\cM}
\newcommand{\cMhat}{\wh{\cM}}

% Core complexity measure

% \newcommand{\comp}[1][\gamma]{\cV_{#1}}
\newcommand{\compbasic}{\mathsf{dec}}
\newcommand{\comp}[1][\gamma]{\mathsf{dec}_{#1}}
\newcommand{\comprand}[1][\gamma]{\mathsf{dec}^{\mathrm{rnd}}_{#1}}
\newcommand{\compshort}[1][\gamma]{\mathsf{d}_{#1}}

\newcommand{\compt}[1][\gamma]{\wt{\mathsf{dec}}_{#1}}

\newcommand{\compbar}[1][\gamma]{\overline{\mathsf{dec}}_{#1}}
\newcommand{\compb}[1][\gamma]{\underline{\mathsf{dec}}_{#1}}
\newcommand{\compdual}[1][\gamma]{\compb[#1]}
\newcommand{\compbayes}[1][\gamma]{\compb[#1]}

% \newcommand{\compKL}{\comp^{\mathrm{KL}}}
% \newcommand{\compKL}{\compgen[\Dklshort]}
% \newcommand{\compKLdual}{\compgendual[\Dklshort]}
\newcommand{\compKL}{\compgen[\mathrm{KL}]}
\newcommand{\compKLdual}{\compgendual[\mathrm{KL}]}

% \newcommand{\compSq}{\compgen[\Dsqshort]}
% \newcommand{\compSqdual}{\compgendual[\Dsqshort]}
\newcommand{\compSq}{\compgen[\mathrm{Sq}]}

\newcommand{\compSqdual}{\compgendual[\mathrm{Sq}]}

\newcommand{\compopt}{\comp^{\mathrm{opt}}}
\newcommand{\compoptdual}{\compb^{\mathrm{opt}}}

\newcommand{\compH}{\compgen[\mathrm{H}]}
\newcommand{\compHdual}{\compgendual[\mathrm{H}]}

\newcommand{\complocshort}{\comp^{\mathrm{loc}}}
\newcommand{\comploc}[2][\gamma]{\mathsf{dec}_{#1,#2}}

\newcommand{\compthmone}{\comploc{\vepslowg}(\cM)}
  \newcommand{\compthmtwo}{\comploc{\vepsupg}(\conv(\cM))}

\newcommand{\comptext}{decision-estimation coefficient\xspace}
\newcommand{\Comptext}{Decision-estimation coefficient\xspace}
\newcommand{\CompText}{Decision-Estimation Coefficient\xspace}
\newcommand{\CompAbbrev}{DEC\xspace}
\newcommand{\CompShort}{\CompAbbrev}

\newcommand{\compgen}[1][D]{\comp^{#1}}
\newcommand{\compgendual}[1][D]{\compb^{#1}}
\newcommand{\compd}[1][D]{\comp^{#1}}

\newcommand{\est}{\mathsf{est}}

\newcommand{\AlgText}{Estimation-to-Decisions\xspace}
\newcommand{\mainalg}{\textsf{E\protect\scalebox{1.04}{2}D}\xspace}
\newcommand{\mainalgsection}{\texorpdfstring{\mainalg}{\textsf{E2D}\xspace}}% Fixes appearance in hyperref bookmarks
% \newcommand{\mainalgsection}{\textsf{E2D}\xspace}
\newcommand{\mainalgL}{\textsf{E\protect\scalebox{1.04}{2}D}$^{\textsf{+}}$\xspace}
\newcommand{\mainalgB}{\textsf{E\protect\scalebox{1.04}{2}D.Bayes}\xspace}

% \newcommand{\expdim}{exploration dimension\xspace}
% \newcommand{\Expdim}{Exploration dimension\xspace}
% \newcommand{\ExpDim}{Exploration Dimension\xspace}

% Superscript for models etc.

\newcommand{\M}[1]{^{{\scriptscriptstyle M}}}  % Small M superscript
\newcommand{\sM}{\sss{M}}
\newcommand{\sMbar}{\sss{\Mbar}}
\renewcommand{\ss}[1]{^{{\scriptscriptstyle#1}}} 
\newcommand{\sups}[1]{^{{\scriptscriptstyle#1}}}
\newcommand{\subs}[1]{_{{\scriptscriptstyle#1}}} 
\renewcommand{\sp}[1]{^{{\scriptscriptstyle#1}}}
\renewcommand{\sb}[1]{_{{\scriptscriptstyle#1}}} 
\newcommand{\sss}[1]{{\scriptscriptstyle#1}}

\newcommand{\Ens}[2]{\En^{\sss{#1},#2}}
\newcommand{\Em}[1][M]{\En^{\sss{#1}}}
\newcommand{\Embar}{\En\sups{\Mbar}}
\newcommand{\Enm}[1][M]{\En^{\sss{#1}}}
\newcommand{\Enmbar}[1][\Mbar]{\En^{\sss{#1}}}
\newcommand{\Prs}[1]{\bbP\sups{#1}}
\newcommand{\bbPs}[1]{\bbP\sups{#1}}
\newcommand{\bbPm}[1][M]{\bbP\sups{#1}}
\newcommand{\bbPmbar}[1][\Mbar]{\bbP\sups{#1}}

% \newcommand{\dm}[1]{d\sups{#1}}

\newcommand{\fmhat}{f\sups{\Mhat}}
% \newcommand{\fmhatt}{f\sups{\Mhat\ind{t}}}

% General setup
\newcommand{\fm}[1][M]{f\sups{#1}}
\newcommand{\pim}[1][M]{\pi_{\sss{#1}}}
\newcommand{\pimh}[1][h]{\pi_{\sss{M,#1}}}

\newcommand{\gm}{g\sups{M}}

\newcommand{\Mprime}{M'}
\newcommand{\pimprime}{\pi_{\sss{M'}}}

\newcommand{\cFm}{\cF_{\cM}}
\newcommand{\Pim}{\Pi_{\cM}}
\newcommand{\cMf}{\cM_{\cF}}

\newcommand{\fmbar}{f\sups{\Mbar}}
\newcommand{\pimbar}{\pi\subs{\Mbar}}

\newcommand{\fmstar}{f\sups{\Mstar}}
\newcommand{\pimstar}{\pi\subs{\Mstar}}

\newcommand{\fstar}{f^{\star}}
\newcommand{\pistar}{\pi^{\star}}

\newcommand{\pihat}{\wh{\pi}}

\newcommand{\cMloc}[1][\veps]{\cM_{#1}}

\newcommand{\Mbar}{\wb{M}}

\newcommand{\fmi}{f\sups{M_i}}
\newcommand{\pimi}{\pi\subs{M_i}}

% RL Setup
\newcommand{\Jm}{J\sups{M}}
\newcommand{\Rm}[1][M]{R\sups{#1}}
\newcommand{\Pm}[1][M]{P\sups{#1}}
\newcommand{\Pmstar}[1][\Mstar]{P\sups{#1}}

\newcommand{\Pmbar}{P\sups{\Mbar}}
\newcommand{\Rmbar}{R\sups{\Mbar}}


\newcommand{\Prm}[2]{\bbP^{\sss{#1},#2}}

\newcommand{\Jmstar}{J\sups{\Mstar}}

\newcommand{\PiNS}{\Pi_{\mathrm{NS}}} % Non-stationary, non-randomized
\newcommand{\PiRNS}{\Pi_{\mathrm{RNS}}} % Non-stationary, non-randomized
\newcommand{\PiNM}{\Pi_{\mathrm{NM}}} %Non-markov, non-randomized
\newcommand{\PiGen}{\Pi_{\mathrm{RNS}}} % Placeholder for "generic" setting.

% \newcommand{\pim}{\pi\ss{M}}

\newcommand{\Qmistar}[1][i]{Q^{#1,\star}}
\newcommand{\Vmistar}[1][i]{V^{#1,\star}}

\newcommand{\Qmstar}[1][M]{Q^{\sss{#1},\star}}
\newcommand{\Vmstar}[1][M]{V^{\sss{#1},\star}}
\newcommand{\Qmpi}[1][\pi]{Q^{\sss{M},#1}}
\newcommand{\Vmpi}[1][\pi]{V^{\sss{M},#1}}
\newcommand{\Qmbarpi}[1][\pi]{Q^{\sss{M},#1}}
\newcommand{\Vmbarpi}[1][\pi]{V^{\sss{M},#1}}
\newcommand{\Qm}[2]{Q^{\sss{#1},#2}}
\newcommand{\Vm}[2]{V^{\sss{#1},#2}}

\newcommand{\dmpi}[1][\pi]{d^{\sss{M},#1}}
\newcommand{\dmbarpi}[1][\pi]{d^{\sss{\Mbar},#1}}
\newcommand{\dm}[2]{d^{\sss{#1},#2}}
\newcommand{\dbar}{\bar{d}}
\newcommand{\dtil}{\wt{d}}

% Regret
\newcommand{\Reg}{\mathrm{\mathbf{Reg}}}
\newcommand{\RegBar}{\widebar{\mathrm{\mathbf{Reg}}}}

\newcommand{\Est}{\mathrm{\mathbf{Est}}}
\newcommand{\EstBar}{\widebar{\mathrm{\mathbf{Est}}}}

\newcommand{\EstHel}{\mathrm{\mathbf{Est}}_{\mathsf{H}}}
\newcommand{\EstOnHel}{\mathrm{\mathbf{Est}}^{\mathsf{on}}_{\mathsf{H}}}
\newcommand{\EstBarHel}{\widebar{\mathrm{\mathbf{Est}}}_{\mathsf{H}}}
\newcommand{\EstProbHel}{\mathrm{\mathbf{Est}}_{\mathsf{H}}(T,\delta)}
\newcommand{\EstH}{\EstHel}

\newcommand{\EstD}{\mathrm{\mathbf{Est}}_{\mathsf{D}}}

\newcommand{\EstSq}{\mathrm{\mathbf{Est}}_{\mathsf{Sq}}}

\newcommand{\EstKL}{\mathrm{\mathbf{Est}}_{\mathsf{KL}}}

% \newcommand{\RegDM}{\Reg_{\mathsf{Dec}}}
% \newcommand{\RegBarDM}{\RegBar_{\mathsf{Dec}}}
\newcommand{\RegDM}{\Reg_{\mathsf{DM}}}
\newcommand{\RegBarDM}{\RegBar_{\mathsf{DM}}}

\newcommand{\RegB}{\RegBandit}
\newcommand{\RegMAB}{\RegBandit}
  \newcommand{\RegBarMAB}{\RegBar_{\mathsf{B}}}
  \newcommand{\RegBandit}{\mathrm{\mathbf{Reg}}_{\mathsf{B}}}

  \newcommand{\RegRL}{\Reg_{\mathsf{RL}}}
\newcommand{\RegBarRL}{\RegBar_{\mathsf{RL}}}


\newcommand{\RegOL}{\mathrm{\mathbf{Reg}}_{\mathsf{OL}}(T,\delta)}
  \newcommand{\AlgOL}{\mathrm{\mathbf{Alg}}_{\mathsf{OL}}}
  \newcommand{\RegOLShort}{\Reg_{\mathsf{OL}}}
  
  \newcommand{\RegBR}{\Reg_{\mathsf{BR}}}
  \newcommand{\RegSB}{\Reg_{\mathsf{SB}}}
  \newcommand{\RegSquare}{\Reg_{\mathsf{Sq}}}
  % \newcommand{\RegCB}{\Reg_{\mathsf{CB}}}
  
  \newcommand{\RegCB}{\mathrm{\mathbf{Reg}}_{\mathsf{CB}}}
  \newcommand{\RegBarCB}{\widebar{\mathrm{\mathbf{Reg}}}_{\mathsf{CB}}}


  %%%%%%%%%%%%%%%%%%%%%%%%%%%%%%%%%%%%%%%%%%%%%%%%%%%%
  % Estimation
  %%%%%%%%%%%%%%%%%%%%%%%%%%%%%%%%%%%%%%%%%%%%%%%%%%%%
  
  \newcommand{\AlgEst}{\mathrm{\mathbf{Alg}}_{\mathsf{Est}}}

  \newcommand{\logloss}{\ls_{\mathrm{\log}}}

  \newcommand{\RegLog}{\Reg_{\mathsf{KL}}}
  \newcommand{\RegBarLog}{\widebar{\Reg}_{\mathsf{KL}}}
    \newcommand{\RegLogT}{\Reg_{\mathsf{KL}}(T)}

    % \newcommand{\MComp}{\kappa(\cM,T)}
    % \newcommand{\Pcomp}{\kappa(\cP,T)}
    % \newcommand{\ActComp}{\kappa(\Pim,T)}
    % \newcommand{\AComp}{\kappa(\Pim,T)}

    \newcommand{\MComp}{\mathsf{est}(\cM,T)}
    \newcommand{\Mcomp}{\mathsf{est}(\cM,T)}
    
    \newcommand{\GComp}{\mathsf{est}(\cG,T)}
    \newcommand{\Gcomp}{\mathsf{est}(\cG,T)}

    \newcommand{\QComp}{\mathsf{est}(\cQm,T)}
    \newcommand{\Qcomp}{\mathsf{est}(\cQm,T)}
    
    \newcommand{\PComp}{\mathsf{est}(\cP,T)}
    \newcommand{\PComph}[1][h]{\mathsf{est}(\cP_{#1},T)}
    
    \newcommand{\ActComp}{\mathsf{est}(\Pim,T)}
    \newcommand{\AComp}{\mathsf{est}(\Pim,T)}
    \newcommand{\Acomp}{\mathsf{est}(\Pim,T)}

\newcommand{\MCov}[1][\veps]{\cN(\cM,#1)}
    \newcommand{\ActCov}[1][\veps]{\cN(\Pim,#1)}
    
%%%%%%%%%%%%%%%%%%%%%%%%%%%%%%%%%%%%%%%%%%%%%%%%%%%% 
% General macros and CB macros %%%%%%%%%%%%%%%%%%%%%%%%%%%%
%%%%%%%%%%%%%%%%%%%%%%%%%%%%%%%%%%%%%%%%%%%%%%%%%%%%

  \newcommand{\mathrmb}[1]{\mathrm{\mathbf{#1}}}

\newcommand{\Mhat}{\wh{M}}
\newcommand{\Mstar}{M^{\star}}

  \newcommand{\bbPl}{\bbP_{\lambda}}
  \newcommand{\bbQl}{\bbQ_{\lambda}}

\newcommand{\cAhat}{\wh{\cA}}

  \newcommand{\ghat}{\wh{g}}

\newcommand{\thetahat}{\wh{\theta}}
\newcommand{\thetastar}{\theta^{\star}}

\newcommand{\tens}{\otimes}

\newcommand{\Rbar}{\wb{R}}

\newcommand{\Pbar}{\widebar{P}}

\newcommand{\starucb}{star hull UCB\xspace}
\newcommand{\starucblong}{star hull upper confidence bound\xspace}

\newcommand{\optionone}{\textsc{Option I}\xspace}
\newcommand{\optiontwo}{\textsc{Option II}\xspace}
% \newcommand{\oracle}{\textsf{Oracle}\xspace}


\newcommand{\Dis}{\mathrm{Dis}}

% \newcommand{\xdist}{\bbP_x}
\newcommand{\xdist}{\cD}
\newcommand{\rdist}{\bbP_{r}}

\newcommand{\algcomment}[1]{\textcolor{blue!70!black}{\small{\texttt{\textbf{//\hspace{2pt}#1}}}}}
\newcommand{\algcommentlight}[1]{\textcolor{blue!70!black}{\transparent{0.5}\small{\texttt{\textbf{//\hspace{2pt}#1}}}}}
\newcommand{\algcommentbig}[1]{\textcolor{blue!70!black}{\footnotesize{\texttt{\textbf{/*
          #1~*/}}}}}
\newcommand{\algcolor}[1]{\textcolor{blue!70!black}{#1}}
\newcommand{\algspace}{\hspace{\algorithmicindent}}

% Regret
% \newcommand{\RegE}{\mathrm{Reg}}
% \newcommand{\RegExp}{\En\brk{\Reg}}
% \newcommand{\Regbar}{\overline{\mathrm{Reg}}}
% \newcommand{\Regh}{\wh{\mathrm{Reg}}}

% Names
\newcommand{\Holder}{H{\"o}lder\xspace}

\newcommand{\Otilde}{\wt{O}}

\newcommand{\midsem}{\,;}

\newcommand{\trn}{\top}
% \newcommand{\trn}{\intercal}
% \newcommand{\trn}{\mathsf{T}}
\newcommand{\pinv}{\dagger}
\newcommand{\psdleq}{\preceq}
\newcommand{\psdgeq}{\succeq}
\newcommand{\psdlt}{\prec}
\newcommand{\psdgt}{\succ}
\newcommand{\approxleq}{\lesssim}
\newcommand{\approxgeq}{\gtrsim}

\newcommand{\opnorm}{\mathsf{op}}

\newcommand{\eigmin}{\lambda_{\mathrm{min}}}
\newcommand{\eigmax}{\lambda_{\mathrm{max}}}

\renewcommand{\tC}{\tilde{\cC}}

\newcommand{\zero}{\mathbf{0}}

\newcommand{\Id}{I}

\newcommand{\bq}{\mathbf{q}}

\newcommand{\xpi}{x^{\pi}}
\newcommand{\phipi}{\phi^{\pi}}

\newcommand{\lw}{\bar{w}}

\newcommand{\bu}{\mb{u}}
\newcommand{\bv}{\mb{v}}
\newcommand{\bz}{\mb{z}}
\newcommand{\bw}{\mb{w}}

\newcommand{\xt}{x_t}
\newcommand{\xto}{x_{t+1}}

\newcommand{\expone}{(1)}
\newcommand{\exptwo}{(2)}

\newcommand{\xl}[1][T]{x_{1:#1}}
\newcommand{\xr}[1][t]{x_{#1:T}}
\newcommand{\ul}[1][T]{u_{1:#1}}
\newcommand{\ur}[1][t]{u_{#1:T}}
\newcommand{\wl}[1][T]{w_{1:#1}}
\renewcommand{\wr}[1][t]{w_{#1:T}}

\newcommand{\fhat}{\wh{f}}
\newcommand{\ahat}{\wh{a}}
% \newcommand{\fhat}{\hat{f}}
\newcommand{\fbar}{\bar{f}}

% \renewcommand{\ind}[1]{^{(#1)}}
% \renewcommand{\ind}[1]{^{{\scriptscriptstyle(#1)}}}
\renewcommand{\ind}[1]{^{{\scriptscriptstyle#1}}}

\newcommand{\xh}{x_h}
\newcommand{\rh}{r_h}
\renewcommand{\rh}{r_h}
\newcommand{\xhp}{x_{h+1}}

% \newcommand{\Ltwonorm}{L_2(\cD)}

% \newcommand{\TimeOracle}{\cT_{\mathsf{oracle}}}

% \newcommand{\bigoh}{\cO}
% \newcommand{\bigoht}{\wt{\cO}}
\newcommand{\bigoh}{O}
\newcommand{\bigoht}{\wt{O}}
\newcommand{\bigom}{\Omega}
\newcommand{\bigomt}{\wt{\Omega}}
\newcommand{\bigthetat}{\wt{\Theta}}

% \newcommand{\indic}{\mathbbm{1}}
\newcommand{\indic}{\mathbb{I}}

% \newcommand{\amax}{\sigma_{\eta}}
% \newcommand{\smax}{\mathrm{smax}_{\eta}}
% \newcommand{\amin}{\sigma_{\eta}}
% \newcommand{\smin}{\mathrm{smin}_{\eta}}

% \newcommand{\emprisk}{\wh{\cR}}
% \newcommand{\gstar}{g^{\star}}
% \newcommand{\Gstar}{\cG^{\star}}

% \newcommand{\var}{\cV}
% \newcommand{\empvar}{\wh{\cV}_n}
% \newcommand{\istar}{i^{\star}}

\newcommand{\overleq}[1]{\overset{#1}{\leq{}}}
\newcommand{\overgeq}[1]{\overset{#1}{\geq{}}}
\newcommand{\overlt}[1]{\overset{#1}{<}}
\newcommand{\overgt}[1]{\overset{#1}{>}}
\newcommand{\overeq}[1]{\overset{#1}{=}}

\renewcommand{\Pr}{\bbP}

\newcommand{\ftilde}{\tilde{f}}
\newcommand{\fcheck}{\check{f}}
\newcommand{\gtilde}{\tilde{g}}
\newcommand{\empcov}{\wh{\cF}}
\newcommand{\Sbad}{S_{\mathrm{bad}}}
\newcommand{\aprime}{a'}
% \newcommand{\covone}{\empcov_{S_1}}
% \newcommand{\covtwo}{\empcov_{S_2}}
% \newcommand{\cov}{\mathsf{c}}

\newcommand{\undertext}[2]{\underbrace{#1}_{\text{#2}}}

\newcommand{\fup}{\bar{f}}
\newcommand{\flow}{\underbar{f}}

\newcommand{\aone}{a\ind{1}}
\newcommand{\atwo}{a\ind{2}}

\newcommand{\cb}{\mathsf{A}}

\newcommand{\poly}{\mathrm{poly}}
\newcommand{\polylog}{\mathrm{polylog}}


\newcommand{\Dsq}[2]{D_{\mathrm{Sq}}\prn*{#1,#2}}
\newcommand{\Dsqshort}{D_{\mathrm{Sq}}}

\newcommand{\kl}[2]{D_{\mathsf{KL}}\prn*{#1\,\|\,#2}}
\newcommand{\Dkl}[2]{D_{\mathsf{KL}}\prn*{#1\,\|\,#2}}
\newcommand{\Dklshort}{D_{\mathsf{KL}}}
\newcommand{\Dhel}[2]{D_{\mathsf{H}}\prn*{#1,#2}}
\newcommand{\Dgen}[2]{D\prn*{#1\dmid{}#2}}
\newcommand{\Dgent}[2]{D\ind{t}\prn*{#1\dmid{}#2}}
\newcommand{\Dhels}[2]{D^{2}_{\mathsf{H}}\prn*{#1,#2}}
\newcommand{\Ddel}[2]{D_{\Delta}\prn*{#1,#2}}
\newcommand{\Dhelshort}{D^{2}_{\mathsf{H}}}
\newcommand{\Dchis}[2]{D_{\chi^2}\prn*{#1\dmid{}#2}}
% \newcommand{\Dhels}[2]{H^2\prn*{#1,#2}}
\newcommand{\Dtris}[2]{D^{2}_{\Delta}\prn*{#1,#2}}
\newcommand{\Dtri}[2]{D_{\Delta}\prn*{#1,#2}}
\newcommand{\Dtv}[2]{D_{\mathsf{TV}}\prn*{#1,#2}}
\newcommand{\Dtvs}[2]{D^2_{\mathsf{TV}}\prn*{#1,#2}}
\newcommand{\dkl}[2]{d_{\mathrm{KL}}\prn*{#1\,\|\,#2}}
\newcommand{\Dphi}[2]{D_{\phi}(#1\;\|\;#2)}
\newcommand{\dphi}[2]{d_{\phi}(#1\;\|\;#2)}
\newcommand{\klr}[2]{D_{\mathrm{re}}(#1\;\|\;#2)}
\newcommand{\tv}[2]{D_{\mathrm{tv}}(#1\;\|\;#2)}
% \newcommand{\breg}[2]{D_{\cR}(#1\;\|\;#2)}

\newcommand{\DhelsX}[3]{D^{2}_{\mathrm{H}}\prn[#1]{#2,#3}}
% \newcommand{\DhelsX}[3]{H^2\prn[#1]{#2,#3}}

\newcommand{\Ber}{\mathrm{Ber}}
\newcommand{\Rad}{\mathrm{Rad}}
\newcommand{\dmid}{\;\|\;}

% \newcommand{\conv}{\mathrm{conv}}
\newcommand{\conv}{\mathrm{co}}
\newcommand{\diam}{\mathrm{diam}}

% \newcommand{\comp}{\mathrm{c}}

\newcommand{\bls}{\mb{\ls}}

\newcommand{\brak}{{\tiny[]}}

\newcommand{\Deltagap}{\Delta_{\mathrm{policy}}}
\newcommand{\Picsc}{\Pi^{\mathrm{csc}}}

\newcommand{\vcdim}{\mathrm{vcdim}}

\newcommand{\adacb}{\textsf{AdaCB}\xspace}
\newcommand{\squarecb}{\textsf{SquareCB}\xspace}
\newcommand{\falcon}{\textsf{FALCON}\xspace}

%%%%%%%%%%%%%%%%%%%%%%%%%%%%%%%%%%%%%%%%%%%%%%%%%%%%%%%%%%%
%%%%%%%%%%%%%%%%%%%%%%%%%%%%%%%%%%%%%%%%%%%%%%%%%%%%%%%%%%%
%%%%%%%%%%%%%%%%%%%%%%%%%%%%%%%%%%%%%%%%%%%%%%%%%%%%%%%%%%%

\newcommand{\Vf}{V}
\newcommand{\Qf}{Q}

\newcommand{\Qhat}{\wh{\Qf}}
% \newcommand{\Qbar}{\widebar{\mathbf{Q}}}
\newcommand{\Vhat}{\wh{\Vf}}
% \newcommand{\Vbar}{\widebar{\mathbf{V}}}

\newcommand{\Vpi}{\Vf^{\pi}}
\newcommand{\Qpi}{\Qf^{\pi}}
% \newcommand{\Vstar}{\Vf^{\star}}
% \newcommand{\Qstar}{\Qf^{\star}}
\newcommand{\Vstar}{V^{\star}}
\newcommand{\Qstar}{Q^{\star}}

\newcommand{\rmax}{\textsf{Rmax}\xspace}
\newcommand{\valueiteration}{\textsf{ValueIteration}\xspace}

\newcommand{\labelme}[1]{\tag{\theequation}\label{#1}}

\newcommand{\Ssafe}{\cS_{\mathrm{safe}}}
\newcommand{\Esafe}{\cE_{\mathrm{safe}}}
\newcommand{\Tsafe}{T_{\mathrm{safe}}}
\newcommand{\unif}{\mathrm{unif}}

\newcommand{\starhull}{\mathrm{star}}

\newcommand{\pescape}{P_{\mathrm{escape}}}
\newcommand{\vepsucb}{\veps_{\mathrm{ucb}}}
\newcommand{\vepsstati}{\veps_{\mathrm{stat,I}}}
\newcommand{\vepsstatii}{\veps_{\mathrm{stat,II}}}
\newcommand{\nstat}{n_{\mathrm{stat}}}
\newcommand{\Phat}{\wh{P}}

\newcommand{\Ffilt}{\mathfrak{F}}
\newcommand{\Eescape}{\cE_{\textrm{escape}}}

\newcommand{\vepshat}{\wh{\veps}}
\newcommand{\vepsbar}{\widebar{\veps}}
\newcommand{\igw}{\textsf{IGW}}
\newcommand{\ssep}{;\,}
% \newcommand{\Reg}{\mathrm{Reg}}
% \newcommand{\xhs}[1][h]{x_{#1}\sim\mu_{#1}}
% \newcommand{\xh}[1][h]{x_{#1}}
\newcommand{\sh}[1][h]{s_{#1}}

\newcommand{\qh}{q_h}
\newcommand{\Qh}{Q_h}
\newcommand{\im}{\ind{m}}
\newcommand{\sep}{\ssep}
\newcommand{\Creg}{C_{\mathrm{reg}}}
\newcommand{\creg}{c_{\mathrm{reg}}}
\newcommand{\Cucb}{C_{\mathrm{ucb}}}
\newcommand{\gammam}{\gamma\im}

\newcommand{\Eopt}{\cE_{\mathrm{opt}}}

\newcommand{\emi}{\psi}
\newcommand{\dirsum}{\oplus}
\newcommand{\piunif}{\pi_{\mathrm{unif}}}
\newcommand{\cPhat}{\wh{\cP}}

\newcommand{\supp}{\mathrm{supp}}

\newcommand{\cPbar}{\wb{\cP}}

% \newcommand{\conf}{\mathsf{conf}}
% \newcommand{\vepssafe}{\veps_{\mathrm{safe}}}
% \renewcommand{\Ssafe}[1][h]{\cS_{h;\mathrm{safe}}}
% \newcommand{\vepstil}{\wt{\veps}}

% \newcommand{\blockalg}{\textsf{RegRL}\xspace}
% \newcommand{\bonusalg}{\textsf{ConfidenceBounds}}
% \newcommand{\Qlbar}{\underbar{\mathbf{Q}}}
% \newcommand{\Vlbar}{\underbar{\mathbf{V}}}
% \newcommand{\vepsclip}{\veps_{\mathrm{clip}}}
% \newcommand{\betaconf}{\beta_{\mathrm{conf}}}
% \newcommand{\betaucb}{\beta_{\mathrm{ucb}}}
% \newcommand{\betafin}{\beta_{T}}
% % \newcommand{\rhoswitch}{\rho_{\mathrm{switch}}}
% \newcommand{\rhoswitch}{\rho_{\mathrm{switch}}}
% \newcommand{\switch}{\textsc{Switch}}
% \newcommand{\false}{\textsc{False}}
% \newcommand{\true}{\textsc{True}}
\newcommand{\Pstar}{P^{\star}}
% \newcommand{\Pbar}{\widebar{P}}
\newcommand{\Plbar}{\underbar{P}}
\newcommand{\Ptil}{\wt{\cP}}
% \newcommand{\Pbar}{\widebar{\cP}}
% \newcommand{\cPbar}{\widebar{\cP}}
% \newcommand{\cPub}{\widebar{\cP}}
% \newcommand{\cPlb}{\underbar{\cP}}
% \newcommand{\Econf}{\cE_{\mathrm{conf}}}
% \newcommand{\Qucb}[1][h]{\Qf_{\mathrm{ucb};#1}}
% \newcommand{\Qlcb}[1][h]{\Qf_{\mathrm{lcb};#1}}
\newcommand{\astar}{a^{\star}}
% \newcommand{\Ef}{\mathbf{E}}
% \newcommand{\Ebar}{\widebar{\mathbf{E}}}
% \newcommand{\Elbar}{\underbar{\mathbf{E}}}
% \newcommand{\Etil}{\wt{\mathbf{E}}}
% \newcommand{\clip}{\texttt{clip}}
% \newcommand{\gap}{\texttt{gap}}
% \newcommand{\gapcheck}{\check{\texttt{gap}}}
% \newcommand{\gapmin}{\texttt{gap}_{\mathrm{min}}}
% \newcommand{\gap}{\Delta}
% \newcommand{\gapcheck}{\check{\gap}}
% \newcommand{\gapmin}{\gap_{\mathrm{min}}}
% \newcommand{\Wbar}{\widebar{\mathbf{W}}}
% \newcommand{\Wlbar}{\underbar{\mathbf{W}}}

% \renewcommand{\betaconf}[1][h]{\beta_{#1}}
% \renewcommand{\betaconf}[1][h]{\beta_{#1}}
\newcommand{\cFhat}{\wh{\cF}}
\newcommand{\cFbar}{\widebar{\cF}}
\newcommand{\Vtil}{\wt{\Vf}}
\newcommand{\Qtil}{\wt{\Qf}}
\newcommand{\ftil}{\tilde{f}}
% \newcommand{\betatil}{\tilde{\beta}}
% \newcommand{\OP}{\textsf{OP}}
% \newcommand{\ET}{\textsf{ET}}
% \newcommand{\cveps}{c_{\veps,\delta;h+1}}

\newcommand{\mathand}{\quad\text{and}\quad}
\newcommand{\mathwhere}{\quad\text{where}\quad}

  % \newcommand{\cov}{\cH_{\infty}}
%   \newcommand{\skel}{\mathrm{skel}}
% \newcommand{\betahat}{\hat{\beta}}

\def\multiset#1#2{\ensuremath{\left(\kern-.3em\left(\genfrac{}{}{0pt}{}{#1}{#2}\right)\kern-.3em\right)}}

% From CB document %%%%%%%%%%%%%%%%%%%%%%%%%%%%%%%%%%%%%

\newcommand{\grad}{\nabla}
\newcommand{\diag}{\mathrm{diag}}

\newcommand{\op}{\mathrm{op}}

\newcommand{\iid}{i.i.d.\xspace}

\renewcommand{\ls}{\ell}

% Related work section %%%%%%%%%%%%%%%%%%%%%%%%

\newcommand{\glcomp}{c}

% \newcommand{\InfB}{\mathcal{I}^{\mathrm{Bayes}}}
% \newcommand{\InfF}{\mathcal{I}^{\mathrm{Freq}}}
\newcommand{\InfB}{\mathcal{I}_{\mathrm{B}}}
\newcommand{\InfF}{\mathcal{I}_{\mathrm{F}}}

\renewcommand{\emptyset}{\varnothing}

\newcommand{\what}{\wh{w}}


% Misc
\newcommand{\pbar}{\bar{p}}
\newcommand{\phat}{\wh{p}}

% Quantized model
\newcommand{\Meps}{M_{\veps}}

%%% Local Variables:
%%% mode: latex
%%% TeX-master: "paper"
%%% End:


  

\let\underbar\undefined
% Implementation of better \overline alternative from
% https://tex.stackexchange.com/questions/16337/can-i-get-a-widebar-without-using-the-mathabx-package.
% DF Note (8/15/20): Added underbar variant. Exactly the same except
% that \overline is replaced by \underline

%\usepackage{amsmath}
\makeatletter
\let\save@mathaccent\mathaccent
\newcommand*\if@single[3]{%
  \setbox0\hbox{${\mathaccent"0362{#1}}^H$}%
  \setbox2\hbox{${\mathaccent"0362{\kern0pt#1}}^H$}%
  \ifdim\ht0=\ht2 #3\else #2\fi
  }
%The bar will be moved to the right by a half of \macc@kerna, which is computed by amsmath:
\newcommand*\rel@kern[1]{\kern#1\dimexpr\macc@kerna}
%If there's a superscript following the bar, then no negative kern may follow the bar;
%an additional {} makes sure that the superscript is high enough in this case:
\newcommand*\widebar[1]{\@ifnextchar^{{\wide@bar{#1}{0}}}{\wide@bar{#1}{1}}}
\newcommand*\underbar[1]{\@ifnextchar_{{\under@bar{#1}{0}}}{\under@bar{#1}{1}}}
%Use a separate algorithm for single symbols:
\newcommand*\wide@bar[2]{\if@single{#1}{\wide@bar@{#1}{#2}{1}}{\wide@bar@{#1}{#2}{2}}}
\newcommand*\under@bar[2]{\if@single{#1}{\under@bar@{#1}{#2}{1}}{\under@bar@{#1}{#2}{2}}}
\newcommand*\wide@bar@[3]{%
  \begingroup
  \def\mathaccent##1##2{%
%Enable nesting of accents:
    \let\mathaccent\save@mathaccent
%If there's more than a single symbol, use the first character instead (see below):
    \if#32 \let\macc@nucleus\first@char \fi
%Determine the italic correction:
    \setbox\z@\hbox{$\macc@style{\macc@nucleus}_{}$}%
    \setbox\tw@\hbox{$\macc@style{\macc@nucleus}{}_{}$}%
    \dimen@\wd\tw@
    \advance\dimen@-\wd\z@
%Now \dimen@ is the italic correction of the symbol.
    \divide\dimen@ 3
    \@tempdima\wd\tw@
    \advance\@tempdima-\scriptspace
%Now \@tempdima is the width of the symbol.
    \divide\@tempdima 10
    \advance\dimen@-\@tempdima
%Now \dimen@ = (italic correction / 3) - (Breite / 10)
    \ifdim\dimen@>\z@ \dimen@0pt\fi
%The bar will be shortened in the case \dimen@<0 !
    \rel@kern{0.6}\kern-\dimen@
    \if#31
      \overline{\rel@kern{-0.6}\kern\dimen@\macc@nucleus\rel@kern{0.4}\kern\dimen@}%
      \advance\dimen@0.4\dimexpr\macc@kerna
%Place the combined final kern (-\dimen@) if it is >0 or if a superscript follows:
      \let\final@kern#2%
      \ifdim\dimen@<\z@ \let\final@kern1\fi
      \if\final@kern1 \kern-\dimen@\fi
    \else
      \overline{\rel@kern{-0.6}\kern\dimen@#1}%
    \fi
  }%
  \macc@depth\@ne
  \let\math@bgroup\@empty \let\math@egroup\macc@set@skewchar
  \mathsurround\z@ \frozen@everymath{\mathgroup\macc@group\relax}%
  \macc@set@skewchar\relax
  \let\mathaccentV\macc@nested@a
%The following initialises \macc@kerna and calls \mathaccent:
  \if#31
    \macc@nested@a\relax111{#1}%
  \else
%If the argument consists of more than one symbol, and if the first token is
%a letter, use that letter for the computations:
    \def\gobble@till@marker##1\endmarker{}%
    \futurelet\first@char\gobble@till@marker#1\endmarker
    \ifcat\noexpand\first@char A\else
      \def\first@char{}%
    \fi
    \macc@nested@a\relax111{\first@char}%
  \fi
  \endgroup
}
\newcommand*\under@bar@[3]{%
  \begingroup
  \def\mathaccent##1##2{%
%Enable nesting of accents:
    \let\mathaccent\save@mathaccent
%If there's more than a single symbol, use the first character instead (see below):
    \if#32 \let\macc@nucleus\first@char \fi
%Determine the italic correction:
    \setbox\z@\hbox{$\macc@style{\macc@nucleus}_{}$}%
    \setbox\tw@\hbox{$\macc@style{\macc@nucleus}{}_{}$}%
    \dimen@\wd\tw@
    \advance\dimen@-\wd\z@
%Now \dimen@ is the italic correction of the symbol.
    \divide\dimen@ 3
    \@tempdima\wd\tw@
    \advance\@tempdima-\scriptspace
%Now \@tempdima is the width of the symbol.
    \divide\@tempdima 10
    \advance\dimen@-\@tempdima
%Now \dimen@ = (italic correction / 3) - (Breite / 10)
    \ifdim\dimen@>\z@ \dimen@0pt\fi
%The bar will be shortened in the case \dimen@<0 !
    \rel@kern{0.6}\kern-\dimen@
    \if#31
      \underline{\rel@kern{-0.6}\kern\dimen@\macc@nucleus\rel@kern{0.4}\kern\dimen@}%
      \advance\dimen@0.4\dimexpr\macc@kerna
%Place the combined final kern (-\dimen@) if it is >0 or if a superscript follows:
      \let\final@kern#2%
      \ifdim\dimen@<\z@ \let\final@kern1\fi
      \if\final@kern1 \kern-\dimen@\fi
    \else
      \underline{\rel@kern{-0.6}\kern\dimen@#1}%
    \fi
  }%
  \macc@depth\@ne
  \let\math@bgroup\@empty \let\math@egroup\macc@set@skewchar
  \mathsurround\z@ \frozen@everymath{\mathgroup\macc@group\relax}%
  \macc@set@skewchar\relax
  \let\mathaccentV\macc@nested@a
%The following initialises \macc@kerna and calls \mathaccent:
  \if#31
    \macc@nested@a\relax111{#1}%
  \else
%If the argument consists of more than one symbol, and if the first token is
%a letter, use that letter for the computations:
    \def\gobble@till@marker##1\endmarker{}%
    \futurelet\first@char\gobble@till@marker#1\endmarker
    \ifcat\noexpand\first@char A\else
      \def\first@char{}%
    \fi
    \macc@nested@a\relax111{\first@char}%
  \fi
  \endgroup
}
\makeatother



% \let\openbox\relax
% \usepackage{pxfonts}
% \let\openbox\relax
% \usepackage{txfonts}


\usepackage{color-edits}
 % \usepackage[suppress]{color-edits}
 \addauthor{df}{ForestGreen}
 \newcommand{\dfc}[1]{\dfcomment{#1}}

 % \usepackage{showlabels}

 % \usepackage[cut]{thmbox}

% \input{theorembox}
 


  % Indent option for \Statex
\makeatletter
\let\OldStatex\Statex
\renewcommand{\Statex}[1][3]{%
  \setlength\@tempdima{\algorithmicindent}%
  \OldStatex\hskip\dimexpr#1\@tempdima\relax}
\makeatother

% \let\vec\undefined
 \usepackage{accents}
 \newcommand{\ubar}[1]{\underaccent{\bar}{#1}}

%%%%%%%%%%%%%%%%%%%%%%%%%%%%%%%%%%%%%%%%%%%%%%
%%%%%%%%%%%%%%%%%%%%%%%%%%%%%%%%%%%%%%%%%%%%%%
%%%%%%%%%%%%%%%%%%%%%%%%%%%%%%%%%%%%%%%%%%%%%%
\let\oldparagraph\paragraph
\newcommand{\paragraphi}[1]{\oldparagraph{\emph{#1}.}}
\renewcommand{\paragraph}[1]{\oldparagraph{#1.}}

%%%%%%%%%%%%%%%%%%%%%%%%%%%%%%%%%%%%%%%%%%%%%% 
%%%%%%%%%%%%%%%%%%%%%%%%%%%%%%%%%%%%%%%%%%%%%%
%%%%%%%%%%%%%%%%%%%%%%%%%%%%%%%%%%%%%%%%%%%%%%
 \newcommand{\Unif}{\mathsf{Unif}}

\newcommand{\phistar}{\phi^{\star}}
\newcommand{\gstar}{g^{\star}}

\newcommand{\meet}{\wedge} 
\newcommand{\nn}{\nonumber}

\renewcommand{\dagger}{\texttt{DAGGER}\xspace}


%%%%%%%%%%%%%%%%%%%%%%%%%%%%%%%%%%%%%%%%%%%%%%
%%%%%%%%%%%%%%%%%%%%%%%%%%%%%%%%%%%%%%%%%%%%%%
%%%%%%%%%%%%%%%%%%%%%%%%%%%%%%%%%%%%%%%%%%%%%%

%\title{}
\title{Self-Improvement and Bootstrapping}
\author{%
%
}


\date{}

\begin{document}
\maketitle

\tableofcontents

\section{Basic self-improvement setup}
Consider the following setup. The true model is $\pistar:\cX\to\Delta(\cY)$. We
initially receive $n$ samples $\cD=\crl*{(x,y)}$ where $x\sim\rho$ and
$y\sim\pistar$, and we fit $\pi\ind{1}
=\argmax_{\pi\in\Pi}\Eh\brk*{\log\pi(y\mid{}x)}$. We can consider
various self-training protocols
\begin{itemize}
\item \textbf{Regularized maximum likelihood}. We repeatedly do the following:
  \begin{itemize}
  \item Generate a dataset $\cD\ind{t}=\crl*{(x,y)}$ of samples with
    $x\sim\rho$ and $y\sim\pi\ind{t}$.
  \item Define
    \begin{align}
      \label{eq:regularized_mle}
      \pi\ind{t+1}=\argmax\crl*{
      \Eh_{\cD\ind{t}}\brk*{\log\pi(y\mid{}x)}
      -\beta\Dkl{\pi}{\pi\ind{t}}
      }.
    \end{align}
  \end{itemize}
  From first-order conditions, the maximizer here should satisfy
  \[
    \frac{\pi\ind{t}(y\mid{}x)}{\pi(y\mid{}x)}
    =     \beta\log\frac{\pi(y\mid{}x)}{\pi\ind{t}(y\mid{}x)}  + \lambda,
  \]
  where $\lambda\in\bbR$ is a Lagrange multiplier.

A version that has nicer first-order conditions (not sure if there's a
closed form yet) is
    \begin{align}
      \label{eq:regularized_mle_chi}
      \pi\ind{t+1}=\argmax\crl*{
      \Eh_{\cD\ind{t}}\brk*{\log\pi(y\mid{}x)}
      -\frac{\beta}{2}\Dchis{\pi}{\pi\ind{t}}
      }.
    \end{align}
This gives $\pi\ind{t}(y\mid{}x)^2 = \beta\pi^2(y\mid{}x) +
2\lambda\pi(y\mid{}x)\pi\ind{t}(y\mid{}x)$ for a Lagrange multiplier
$\lambda\in\bbR$. Solving the quadratic equation then gives
$\pi = \pi\ind{t}\cdot(\pm\sqrt{\lambda^2+\beta^2}-\lambda)/\beta$,
and setting $\lambda$ so that this normalizes to $1$ just gives $\pi\ind{t+1}=\pi\ind{t}$.
\item \textbf{Reverse maximum likelihood (Akshay's objective)}.
  We repeatedly do the following:
  \begin{itemize}
  \item Generate a dataset $\cD\ind{t}=\crl*{(x,y)}$ of samples with
    $x\sim\rho$ and $y\sim\pi\ind{t}$.
  \item Define
    \begin{align}
      \label{eq:self_training}
      \pi\ind{t+1}=\argmax\crl*{
      \En_{\pi}\brk*{\log\pi\ind{t}(y\mid{}x)}
      -\beta\Dkl{\pi}{\pi\ind{t}}
      }.
    \end{align}
  \end{itemize}
  With no errors, this converges to
  \[
    \pi\ind{t+1}(y\mid{}x)\propto{}\pistar(y\mid{}x)^{(1+\beta^{-1})^{T}}.
  \]
Here's an observation: If we define the entropy as
$H(\pi)=-\En_{x\sim\rho{},y\sim\pi}\brk*{\log\pi(y\mid{}x)}$, when
$\grad_{\pi}H=-\log(\pi(\cdot\mid{}\cdot)) + \mb{1}$. Hence, we can
interpret \cref{eq:self_training} as doing exponentiated gradient
\emph{ascent} on the convex objective
\[
R(\pi) = -H(\pi).
\]
That is, this is an instance of non-concave maximization, but the
local optimum is the argmax policy $\argmax_{y}\pistar(y\mid{}x)$.
  
  \item \textbf{Regularized maximum likelihood with reverse-KL regularization}. We repeatedly do the following:
  \begin{itemize}
  \item Generate a dataset $\cD\ind{t}=\crl*{(x,y)}$ of samples with
    $x\sim\rho$ and $y\sim\pi\ind{t}$.
  \item Define
    \begin{align}
      \label{eq:regularized_mle_reverse}
      \pi\ind{t+1}=\argmax\crl*{
      \Eh_{\cD\ind{t}}\brk*{\log\pi(y\mid{}x)}
      -\beta\Dkl{\pi\ind{t}}{\pi}
      }.
    \end{align}
  \end{itemize}
  \dfc{This is equivalent to just
    $\argmax_{\pi}(1+\beta)\En_{\pi\ind{t}}\brk*{\log\pi(y\mid{}x)}$,
    which means $\beta$ has no effect. Probably not the best choice}
\end{itemize}
\cref{eq:regularized_mle} is kind of like iterative SFT with
pseudo-labels, whereas \cref{eq:self_training} is closer to iterative
RLHF with pseudo-labels.
\subsection{Questions and comments}
We know that with no statistical or optimization errors, the estimator
in \cref{eq:self_training} should converge to
  \[
    \pi\ind{t+1}(y\mid{}x)\propto{}\pistar(y\mid{}x)^{(1+\beta^{-1})^{T}}
  \]
  % \dfc{Check this!}
  Questions:
\begin{itemize}
\item When can we achieve similar convergence in the presence of
  statistical errors? E.g., what are the tradeoffs for choosing $\beta$?
\item What are the different tradeoffs for \cref{eq:regularized_mle,eq:self_training,eq:regularized_mle_reverse}?
\item Is this purely a computational phenomenon?
  \begin{itemize}
  \item Concretely: Is $\pi\ind{t+1}$ ever better statistically than
    computing the sequence-level argmax
    $\argmax_{y}\pi\ind{1}(y\mid{}x)$? If not, then we want to argue
    that the benefit is that we are competing with the sequence level
    argmax without having to actually compute it (that is, given a new
    $x$, we just need to sample $y\sim{}\pi\ind{t+1}(y\mid{}x)$.
    \begin{itemize}
    \item This would be a sort of ``optimization vs sampling'' phenomenon.
    \end{itemize}
  \item Alternatively, we might be able to argue that our algo *does*
    have statistical benefits, because it allows us to leverage the
    unlabeled examples $x$s.
  \end{itemize}
\item Presumably we need some kind of margin condition to ensure that
  \cref{eq:self_training} actually converges to argmax instead of
  noise. This is most easy to think about when $\pistar$ is itself a
  softmax policy.
\item Suppose $\pi\ind{1}$ is an LLM with temperature parameter
  $\eta$. We have $y=y_{1:H}$, and
  $\pi(y_h\mid{}x,y_{1:h-1})\propto\exp\prn*{\frac{f(x,y_{1:h})}{\eta}}$. A
  good baseline for us is to set $\eta\to{}0$ after training. In this
  case, sampling autoregressively is the same as choosing the greedy
  argmax at each step $h$,
  i.e. $y_h=\argmax{}\pi(y_h\mid{}x,y_{1:h-1})$, but is not
  necessarily equivalent to the sequence-level argmax.
  \begin{itemize}
  \item Hence, we might hope that self-training can improve over this
    baseline, even if it can't improve over the true sequence-level
    argmax.
  \item Good example to think about above: $f=\Qstarb$ is the soft
    Q-function for a given reward function $r$.
  \end{itemize}
\item In reality, we will not be able to evaluate the KL term in
  \cref{eq:self_training} exactly, due to statistical errors. Does the
  objective still work when we have statistical errors on this term?
  Moreover, can we reduce the optimization to a sequence of linear
  opt/policy opt-style problems via frank-wolfe?
\item What realizability assumptions do we actually need in order for
  \cref{eq:self_training} to work out?
\end{itemize}

\section{Finite-Sample Analysis}

\dfc{
  \begin{itemize}
  \item Gradient dominance
  \item Best-of-k baseline. Also, explain what margin condition
    entails for softmax.
  \item Exp weights and convergence?
    \begin{itemize}
    \item Interesting idea: Assume $\veps$-apx update even though we
      have no statistical errors.
    \item Reducing objective to linear optimization via frank-wolfe?
    \end{itemize}
  \end{itemize}
  }

  \subsection{Non-Contextual Case}
  \newcommand{\sm}{\alpha}
  \newcommand{\ystar}{y^{\star}}
  \newcommand{\ones}{\mb{1}}
  \newcommand{\pia}{\pi_{\alpha}}
In this section, we analyze \cref{eq:self_training} in the
non-contextual case where $\cX=\emptyset$. Let a parameter
$\sm\in(0,1/2)$ and policy $\piref$ be given. Define
\[
  \pi_\alpha = (1-\alpha)\pi + \alpha\piref.
\]
We consider the following smoothed version
of the update in \cref{eq:self_training}:
\begin{align}
  \pi\ind{t+1}=\argmax_{\pi}\crl*{
  \En_{\pi}\brk*{\log\pi_\alpha(y)}
  -\beta(1-\alpha)^{-1}\Dkl{\pi_\alpha}{\pi_\alpha\ind{t}}
  }.
\end{align}
To make things more interesting, we consider the following inexact
update. Defining
\[
\Psi\ind{t}(\pi) =     \En_{\pi}\brk*{\log\pi_\alpha(y)}
-\beta(1-\alpha)^{-1}\Dkl{\pi_\alpha}{\pi_\alpha\ind{t}},
% \En_{\pi}\brk*{\log\prn*{(1-\alpha)\pi\ind{t}(y)+\alpha\piref(y)}}
%   -\beta\Dkl{\pi}{\pi\ind{t}},
\]
we assume that
\begin{align}
  \label{eq:pit}
\Psi\ind{t}(\pi\ind{t+1})\geq{}\max_{\pi}\Psi\ind{t}(\pi) - \veps
\end{align}
for some $\veps\geq{}0$.
\begin{assumption} We assume that:
  \begin{itemize}
  \item For all $t$, $\pi\ind{t}\in\Pi$ for some known class $\Pi$.
  \item All $\pi\in\Pi$ satisfy $\nrm*{\frac{\pi}{\piref}}_{\infty}\leq{}\Cconc$.
  \end{itemize}
  
\end{assumption}

\begin{theorem}
  \label{thm:bandit}
  
\end{theorem}

\begin{proof}[\pfref{thm:bandit}]
  Define the smoothed entropy via
  \begin{align}
    \label{eq:smoothed_entropy}
    H_\alpha(\pi) = (1-\alpha)^{-1}H((1-\alpha)\pi+\alpha\piref) = (1-\alpha)^{-1}H(\pi_\alpha),
  \end{align}
  where $H(\pi)\ldef{}\sum_{y}\pi(y)\log\prn*{\frac{1}{\pi(y)}}$.
  \begin{lemma}
    \label{lem:entropy_gradient}
    We have that
    $\grad{}H_\alpha(\pi) =
      -\prn*{\log((1-\alpha)\pi(\cdot)+\alpha\piref(\cdot)) + \ones} =
\grad{}H(\pi)\mid_{\pi=\pia}$.
    \end{lemma}
    Let $\pibar\ind{t}\ldef{}(1-\alpha)\pi\ind{t}+\alpha\piref=\pi_\alpha\ind{t}$.
    The following result will be the starting point for our analysis.
    \begin{lemma}
      \label{lem:first_order}
      For all $\pistar$, we have that
      \begin{align*}
        \beta^{-1}\tri*{\grad{}H(\pia\ind{t}),\pia\ind{t+1}-\pistar}
        \leq{} \Dkl{\pistar}{\pia\ind{t}}
  - \Dkl{\pistar}{\pia\ind{t+1}}
  - \Dkl{\pia\ind{t+1}}{\pia\ind{t}} + \beta^{-1}\cdot\underbrace{\prn*{\veps + 2\alpha\log(\abs{\cY}\alpha^{-1})}}_{\rdef\veps'}.
      \end{align*}
    \end{lemma}
    Let
    $\ystar\ldef{}\argmax_{y\in\cY}\pibar\ind{1}(y)$. We
    assume that for some $\gamma<1$,
    \begin{align*}
      \pibar\ind{1}(\ystar)\geq{}(\pibar\ind{1}(y))^{\gamma}\quad\forall{}y\neq\ystar.
    \end{align*}
    Our goal is to prove inductively that for all $t\in\brk{T}$,
    \dfc{conditions on lr and error}
    \begin{align}
      \label{eq:ind1}
      \pibar\ind{t}(\ystar)\geq{}(\pibar\ind{t}(y))^{\gamma}\quad\forall{}y\neq\ystar,
        \end{align}
        and
        \begin{align}
          \label{eq:ind2}
          \dfc{TBD}
        \end{align}
        \paragraph{Inductive step}
        Let $t\in\brk{T}$ be fixed, and assume that
        $\pi\ind{1},\dots,\pi\ind{t}$ satisfy
        \cref{eq:ind1,eq:ind2}. Our goal will be to show that
        $\pi\ind{t+1}$ satisfies the same property. If we set
        $\pistar=e_{\ystar}$, then rearranging
        \cref{lem:first_order} implies that
        \begin{align*}
          \log\prn*{
          \frac{\pibar\ind{t}(\ystar)}{\pibar\ind{t+1}(\ystar)}
          }
          &= \Dkl{\pistar}{\pibar\ind{t+1}}-\Dkl{\pistar}{\pibar\ind{t}}\\
            &\leq{}
          -\beta^{-1}\tri*{\grad{}H(\pibar\ind{t}),\pibar\ind{t+1}-\pistar}
              - \Dkl{\pibar\ind{t+1}}{\pibar\ind{t}} + \beta^{-1}\veps'\\
          &=
            -\beta^{-1}\tri*{\grad{}H(\pibar\ind{t}),\pibar\ind{t}-\pistar}
            + \beta^{-1}\tri*{\grad{}H(\pibar\ind{t}),\pibar\ind{t}-\pibar\ind{t+1}}
          - \Dkl{\pibar\ind{t+1}}{\pibar\ind{t}} + \beta^{-1}\veps'.
        \end{align*}
        We appeal to two lemmas to bound the terms on the right-hand side.
        \begin{lemma}[Gradient dominance]
          \label{lem:entropy_gd}
          If $\pi\ind{t}$ satisfies \cref{eq:ind1}, then
          \begin{align*}
            \tri*{\grad{}H(\pibar\ind{t}),\pibar\ind{t}-e_{\ystar}}
            \geq{}
            \prn*{\frac{1}{\gamma}-1}\prn*{1-\pibar\ind{t}(\ystar)}\log\prn*{\frac{1}{\pibar\ind{t}(\ystar)}}.
            % - \alpha\log\prn*{\abs{\cY}\alpha^{-1}}.
            % \prn*{\pibar\ind{t}(\ystar)  + \frac{1-\pibar\ind{t}(\ystar)}{\gamma}-1}\log\prn*{\frac{1}{\pibar\ind{t}(\ystar)}}- \alpha\log\prn*{\abs{\cY}\alpha^{-1}}
          \end{align*}
\dfc{this gives a nontrivial lower bound even when $\gamma=1$, which
  is slightly sus. maybe this is fine though, since a $\geq{}0$ lower
  bound is not enough to imply a concrete rate of convergence}          
\end{lemma}
In particular, if we define $\eta =
\frac{\beta^{-1}(1-\gamma)}{2\gamma}$, then as long as
$\pistar\ind{t}(y)\leq{}1/2$, this implies that
        \begin{align*}
          \log\prn*{
          \frac{\prn*{\pibar\ind{t}(\ystar)}^{1-\eta}}{\pibar\ind{t+1}(\ystar)}
          }
\leq
          \beta^{-1}\tri*{\grad{}H(\pibar\ind{t}),\pibar\ind{t}-\pibar\ind{t+1}}
          - \Dkl{\pibar\ind{t+1}}{\pibar\ind{t}} + \beta^{-1}\veps'.
        \end{align*}
        Next, we use the following lemma.
        \begin{lemma}
          \label{lem:err_bound}
          For all $\zeta>0$,
          \begin{align*}
            \tri*{\grad{}H(\pibar\ind{t}),\pibar\ind{t}-\pibar\ind{t+1}}
            \leq \zeta{}\cdot{}e^2 \log^2(\Cconc\abs{\cY}\alpha^{-1})
            + \zeta^{-1}\cdot{}\min\crl*{\Dkl{\pibar\ind{t+1}}{\pibar\ind{t}}, \Dkl{\pibar\ind{t}}{\pibar\ind{t+1}}}.
          \end{align*}
        \end{lemma}
        With this, we conclude that
        \begin{align*}
          \log\prn*{
          \frac{\prn*{\pibar\ind{t}(\ystar)}^{1-\eta}}{\pibar\ind{t+1}(\ystar)}
          }
          \leq \beta^{-2}\underbrace{e^2 \log^2(\Cconc\abs{\cY}\alpha^{-1})}_{\rdef{}C}
          + 
\beta^{-1}\veps',
        \end{align*}
        or
        \begin{align*}
          \pibar\ind{t+1}(\ystar) \geq{} \prn{\pibar\ind{t}(\ystar)}^{1-\eta}\cdot{}e^{-(\beta^2C+\beta\veps')}.
        \end{align*}
\dfc{Idea for completing the proof: Since above holds for $\ystar$,
  and since $\ystar$ maximizes $\pibar\ind{t}$, it should imply that
  \cref{eq:ind1} holds for $\pibar\ind{t+1}$ as well.}
\dfc{Note quite: Can we also show that
  $\pibar\ind{t}(y)\geq{}c\cdot\pibar\ind{t+1}(y)$ for
  $y\neq{}\ystar$? Then above would be sufficient.
}
\dfc{Possibilities:
  \begin{itemize}
  \item Use
    $H(\pi\ind{t})-C\ind{t}\log\prn*{\frac{1}{\pi\ind{t}(\ystar)}}$ as
    a potential and argue that we can
    $H(\pi\ind{t})-C\ind{t}\log\prn*{\frac{1}{\pi\ind{t}(\ystar)}}\geq{}0$
    for $C\ind{t}$ non-decreasing. This is implied by our condition
    for $C\ind{t}\approx{}\gamma^{-1}$.
    \begin{itemize}
    \item Issue: Cannot deduce from this that $\ystar$ is argmax for
      $\pibar\ind{t}$ early on.
    \end{itemize}
  \item Is it possible that this is not actually true? E.g., unless
    $\veps=\poly(\abs{\cY}^{-1})$ or
    $\beta^{-1}=\poly(\abs{\cY}^{-1})$, argmax can flip?
  \end{itemize}
  }


\begin{lemma}
  \label{lem:entropy_lower}
  If $\ystar=\argmax_{y\in\cY}\pibar\ind{t}(y)$, then
  \begin{align*}
    H(\pibar\ind{t+1}) \geq \log\prn*{\frac{1}{\pibar\ind{t}(\ystar)}}
    - \beta^{-2}\underbrace{e^2
    \log^2(\Cconc\abs{\cY}\alpha^{-1})}_{\rdef{}C}
    -\beta^{-1}\veps'.
  \end{align*}
  
\end{lemma}
\begin{proof}[\pfref{lem:entropy_lower}]
  Applying \cref{lem:first_order} with $\pistar=\pibar\ind{t}$, we
  have that
  \begin{align*}
    \Dkl{\pibar\ind{t+1}}{\pibar\ind{t}}
    \leq
    \beta^{-1}\tri*{\grad{}H(\pibar\ind{t}),\pibar\ind{t}-\pibar\ind{t+1}}
    -\Dkl{\pibar\ind{t}}{\pibar\ind{t+1}} + \beta^{-1}\veps'
    \leq{} \beta^{-2}C + \beta^{-1}\veps'
    % \underbrace{e^2 \log^2(\Cconc\abs{\cY}\alpha^{-1})}_{\rdef{}C}
  \end{align*}
  by \cref{lem:err_bound}. We then note that
  \begin{align*}
    \Dkl{\pibar\ind{t+1}}{\pibar\ind{t}}
    &22= \sum_{y}\pibar\ind{t+1}(y)\log\prn*{\frac{1}{\pibar\ind{t}(y)}}
    - H(\pibar\ind{t+1})\\
    &\geq{} \sum_{y}\pibar\ind{t+1}(y)\log\prn*{\frac{1}{\pibar\ind{t}(\ystar)}}
    - H(\pibar\ind{t+1})
    =\log\prn*{\frac{1}{\pibar\ind{t}(\ystar)}}
    - H(\pibar\ind{t+1})
  \end{align*}
  as long as $y\in\argmax_{y\in\cY}\pibar\ind{t}(y)$.
\end{proof}

\begin{lemma}
  \label{lem:entropy_log}
  If $\ystar=\argmax_{y\in\cY}\pibar\ind{t}(y)$, then
  \begin{align*}
    \log\prn*{\frac{1}{\pibar\ind{t+1}(\ystar)}}
    -    \log\prn*{\frac{1}{\pibar\ind{t}(\ystar)}}
    \leq \beta^{-1}\prn*{H(\pibar\ind{t})-H(\pibar\ind{t+1})} + \beta^{-1}\veps'.
  \end{align*}
\end{lemma}
\begin{proof}[\pfref{lem:entropy_log}]
  From \cref{lem:first_order} with $\pistar=e_{\ystar}$, we have that
  \begin{align*}
    \log\prn*{\frac{1}{\pibar\ind{t+1}(\ystar)}}
    -    \log\prn*{\frac{1}{\pibar\ind{t}(\ystar)}}
    &\leq -
    \beta^{-1}\tri*{\grad{}H(\pibar\ind{t}),\pibar\ind{t+1}-e_{\ystar}}
      -\Dkl{\pibar\ind{t+1}}{\pibar\ind{t}} + \beta^{-1}\veps'\\
        &\leq -
    \beta^{-1}\tri*{\grad{}H(\pibar\ind{t}),\pibar\ind{t+1}-e_{\ystar}}
          +\beta^{-1}\veps'.
  \end{align*}
  We observe that
  \begin{align*}
    -\tri*{\grad{}H(\pibar\ind{t}),\pibar\ind{t+1}-e_{\ystar}}
    &= \log\prn*{\frac{1}{\pibar\ind{t}(\ystar)}}
    -
    \sum_{y\in\cY}\pibar\ind{t+1}(y)\log\prn*{\frac{1}{\pibar\ind{t}(y)}}\\
    &= \log\prn*{\frac{1}{\pibar\ind{t}(\ystar)}}
    -
    \sum_{y\in\cY}\pibar\ind{t+1}(y)\log\prn*{\frac{1}{\pibar\ind{t+1}(y)}}
    -
      \sum_{y\in\cY}\pibar\ind{t+1}(y)\log\prn*{\frac{\pibar\ind{t+1}}{\pibar\ind{t}(y)}}\\
    &\leq \log\prn*{\frac{1}{\pibar\ind{t}(\ystar)}}
    -
      H(\pibar\ind{t+1})
      -  \Dkl{\pibar\ind{t+1}}{\pibar\ind{t}}\\
        &\leq \log\prn*{\frac{1}{\pibar\ind{t}(\ystar)}}
    -
      H(\pibar\ind{t+1}).
  \end{align*}
  Finally, we note that if
  $\ystar\in\argmax_{y\in\cY}\pibar\ind{t}(y)$, then
  \begin{align*}
    \log\prn*{\frac{1}{\pibar\ind{t}(\ystar)}}
    =
    \sum_{y}\pibar\ind{t}(y)\log\prn*{\frac{1}{\pibar\ind{t}(\ystar)}}
    \leq{}     \sum_{y}\pibar\ind{t}(y)\log\prn*{\frac{1}{\pibar\ind{t}(y)}}=H(\pibar\ind{t}).
  \end{align*}
  
\end{proof}

  \end{proof}

  \begin{proof}[\pfref{lem:entropy_gradient}]
    Direct calculation.
  \end{proof}
  
      \begin{proof}[\pfref{lem:first_order}]
      Let
      $\Phi\ind{t}(\pi)\ldef{}\tri*{\pi,\grad{}H_\alpha(\pi\ind{t})} +
      \beta(1-\alpha)^{-1}\Dkl{\pia}{\pia\ind{t}}$. Then from \cref{eq:pit}, we have
      that
      \[
        \Phi\ind{t}(\pi\ind{t+1})\leq{}\min_{\pi}\Phi\ind{t}(\pi) + \veps.
      \]
      Since $\Phi\ind{t}$ is convex, this implies that for all
      policies $\pistar$,
      \begin{align}
        \tri*{\Phi\ind{t}(\pi\ind{t+1}),\pi\ind{t+1}-\pistar}
        \leq{} \Phi\ind{t}(\pi\ind{t+1})-\Phi\ind{t}(\pistar) \leq \veps.
      \end{align}
      Noting that $\Dkl{\cdot}{\cdot}$ is a Bregman divergence with
      regularizer $\cR(\pi)\ldef{}-H(\pi)$, we can compute that
      \begin{align*}
        \grad (\pi\mapsto{}\Dkl{\pia}{\pia\ind{t}})
        = (1-\alpha)\prn*{\grad\cR(\pia)-\grad\cR(\pia\ind{t})}.
      \end{align*}
      With \cref{lem:entropy_gradient}, this implies that
      \[
        \tri*{\grad{}H(\pia\ind{t}),\pi\ind{t+1}-\pistar}
        +
        \beta\tri*{\grad\cR(\pia\ind{t+1})-\grad\cR(\pia\ind{t}),\pi\ind{t+1}-\pistar}
        \leq \veps,
      \]
      or,
            \[
        \tri*{\grad{}H(\pia\ind{t}),\pi\ind{t+1}-\pistar}
        \leq
        \beta\tri*{\grad\cR(\pia\ind{t})-\grad\cR(\pia\ind{t+1}),\pi\ind{t+1}-\pistar}
+ \veps.
\]
We can further write this as
\begin{align*}
  \tri*{\grad{}H(\pia\ind{t}),\pia\ind{t+1}-\pistar}
  &\leq
    \beta\tri*{\grad\cR(\pia\ind{t})-\grad\cR(\pia\ind{t+1}),\pia\ind{t+1}-\pistar}
    + \veps\\
  &~~~~+
    \beta\tri*{\grad\cR(\pia\ind{t})-\grad\cR(\pia\ind{t+1}),\pi\ind{t+1}-\pia\ind{t+1}}
    +         \tri*{\grad{}H(\pia\ind{t}),\pia\ind{t+1}-\pi\ind{t+1}},
\end{align*}
and, the Pythagorean inequality for Bregman divergences implies
that
\begin{align*}
  \tri*{\grad\cR(\pia\ind{t})-\grad\cR(\pia\ind{t+1}),\pia\ind{t+1}-\pistar}
  = \Dkl{\pistar}{\pia\ind{t}}
  - \Dkl{\pistar}{\pia\ind{t+1}}
  - \Dkl{\pia\ind{t+1}}{\pia\ind{t}}.
\end{align*}
Finally, we can bound
\begin{align*}
  \tri*{\grad\cR(\pia\ind{t})-\grad\cR(\pia\ind{t+1}),\pi\ind{t+1}-\pia\ind{t+1}}
  &=
    \alpha\sum_{y}(\pi\ind{t+1}(y)-\piref(y))\log\prn*{\frac{\pia\ind{t}(y)}{\pia\ind{t+1}(y)}}\\
  &\leq
    \alpha\log\prn*{\sum_{y}\frac{\pi\ind{t+1}(y)\pia\ind{t}(y)}{\pia\ind{t+1}(y)}}
    +
    \alpha\log\prn*{\sum_y\frac{\piref(y)\pia\ind{t+1}(y)}{\pia\ind{t}(y)}}\\
    &\leq
      \alpha\log\prn*{\alpha^{-1}(1-\alpha)^{-1}}
      \leq{}\alpha\log(2\alpha^{-1}).
\end{align*}
and
\begin{align*}
  \tri*{\grad{}H(\pia\ind{t}),\pia\ind{t+1}-\pi\ind{t+1}}
  &=
    \alpha\sum_{y}(\piref(y)-\pi\ind{t+1}(y))\log\prn*{\frac{1}{\pia\ind{t}(y)}}\\
  &\leq \alpha\sum_{y}\piref(y)\log\prn*{\frac{1}{\pia\ind{t}(y)}}\\
  &\leq \alpha\log\prn*{\sum_{y}\frac{\piref(y)}{\pia\ind{t}(y)}}
    \leq{}\alpha\log\prn*{\abs{\cY}\alpha^{-1}}
\end{align*}
As long as $\abs{\cY}\geq{}2$, we can bound
$\alpha\log(2\alpha^{-1})+\alpha\log\prn*{\abs{\cY}\alpha^{-1}}
\leq{}2\alpha\log(\abs{\cY}\alpha^{-1})$.
\end{proof}

        \begin{proof}[\pfref{lem:entropy_gd}]
          % Using \cref{lem:entropy_gradient}, we can write
          We can write
          \begin{align*}
            \tri*{\grad{}H(\pibar\ind{t}),\pibar\ind{t}-e_{\ystar}}
            % &=
            %   \sum_{y}\pi\ind{t}(y)\log\prn*{\frac{1}{\pibar\ind{t}(y)}}
            %   - \log\prn*{\frac{1}{\pibar\ind{t}(\ystar)}}\\
            % &=
            %   \sum_{y}\pibar\ind{t}(y)\log\prn*{\frac{1}{\pibar\ind{t}(y)}}
            %   - \log\prn*{\frac{1}{\pibar\ind{t}(\ystar)}}
            %   + \alpha\sum_{y}(\pi\ind{t}(y)-\piref(y))
            %   \log\prn*{\frac{1}{\pibar\ind{t}(y)}}\\
            %             &\geq
            %   \sum_{y}\pibar\ind{t}(y)\log\prn*{\frac{1}{\pibar\ind{t}(y)}}
            %   - \log\prn*{\frac{1}{\pibar\ind{t}(\ystar)}}
            %               -
            %               \alpha\sum_{y}\piref(y)\log\prn*{\frac{1}{\pibar\ind{t}(y)}}\\
                                    &=
              \sum_{y}\pibar\ind{t}(y)\log\prn*{\frac{1}{\pibar\ind{t}(y)}}
              - \log\prn*{\frac{1}{\pibar\ind{t}(\ystar)}}
                                      % - \alpha\log\prn*{\abs{\cY}\alpha^{-1}},
          \end{align*}
          % where the last line uses that
          % $\sum_{y}\piref(y)\log\prn*{\frac{1}{\pibar\ind{t}(y)}}
          % \leq{}\log\prn*{\sum_{y}\frac{\piref(y)}{\pibar\ind{t}(y)}}\leq\log(\abs{\cY}\alpha^{-1})$.
          To proceed, using \cref{eq:ind1}, we bound
          \begin{align*}
            \log\prn*{\frac{1}{\pibar\ind{t}(\ystar)}}
            =
            \frac{1}{1-\pibar\ind{t}(\ystar)}\sum_{y\neq\ystar}\pibar\ind{t}(y)
            \log\prn*{\frac{1}{\pibar\ind{t}(\ystar)}}\\
            \leq{}
            \frac{1}{1-\pibar\ind{t}(\ystar)}\sum_{y\neq\ystar}\pibar\ind{t}(y)
            \log\prn*{\frac{1}{(\pibar\ind{t}(y))^{\gamma}}}\\
            = \frac{\gamma}{1-\pibar\ind{t}(\ystar)}\sum_{y\neq\ystar}\pibar\ind{t}(y) \log\prn*{\frac{1}{\pibar\ind{t}(y)}}.
          \end{align*}
          Using this, we have that
          \begin{align*}
            \sum_{y}\pibar\ind{t}(y)\log\prn*{\frac{1}{\pibar\ind{t}(y)}}
            - \log\prn*{\frac{1}{\pibar\ind{t}(\ystar)}}
            &\geq{} \prn*{\pibar\ind{t}(\ystar)  +
            \frac{1-\pibar\ind{t}(\ystar)}{\gamma}-1}\log\prn*{\frac{1}{\pibar\ind{t}(\ystar)}}\\
            &= \prn*{\frac{1}{\gamma}-1}\prn*{1-\pibar\ind{t}(\ystar)}\log\prn*{\frac{1}{\pibar\ind{t}(\ystar)}},
          \end{align*}
          which proves the result.

        \end{proof}

                \begin{proof}[\pfref{lem:err_bound}]
          We can write
          \begin{align*}
            \tri*{\grad{}H(\pibar\ind{t}),\pibar\ind{t}-\pibar\ind{t+1}}
            &=
            \sum_{y}(\pibar\ind{t}(y)-\pibar\ind{t+1}(y))\log\prn*{\frac{1}{\pibar\ind{t}(y)}}\\
            &=
\sum_{y}(\pibar\ind{t}(y)-\pibar\ind{t+1}(y))\log\prn*{\frac{1}{\pibar\ind{t}(y)}}\cdot\frac{\sqrt{\pibar\ind{t}(y)+\pibar\ind{t+1}(y)}}{\sqrt{\pibar\ind{t}(y)+\pibar\ind{t+1}(y)}}\\
            &\leq
              \frac{\zeta}{2}\sum_{y}(\pibar\ind{t}(y)+\pibar\ind{t+1}(y))
              \log^2\prn*{\frac{1}{\pibar\ind{t}(y)}}
              +\frac{\zeta^{-1}}{2}
              \sum_{y}\frac{(\pibar\ind{t}(y)-\pibar\ind{t+1}(y))^2}{\pibar\ind{t}(y)+\pibar\ind{t+1}(y)}\\
                        &\leq
              \frac{\zeta}{2}\sum_{y}(\pibar\ind{t}(y)+\pibar\ind{t+1}(y))
              \log^2\prn*{\frac{1}{\pibar\ind{t}(y)}}
              +\zeta^{-1}\min\crl*{\Dkl{\pibar\ind{t+1}}{\pibar\ind{t}}, \Dkl{\pibar\ind{t}}{\pibar\ind{t+1}}}.
          \end{align*}
          where the last line uses that $
          \sum_{y}\frac{(\pibar\ind{t}(y)-\pibar\ind{t+1}(y))^2}{\pibar\ind{t}(y)+\pibar\ind{t+1}(y)}
          \leq 2
          \sum_{y}(\sqrt{\pibar\ind{t}(y)}-\sqrt{\pibar\ind{t+1}(y)})^2=2\Dhels{\pibar\ind{t+1}}{\pibar\ind{t}}
          \leq2\min\crl*{\Dkl{\pibar\ind{t+1}}{\pibar\ind{t}}, \Dkl{\pibar\ind{t}}{\pibar\ind{t+1}}}$.

Next, we observe that $x\mapsto{}\log^2(x)$ is concave over $x\geq{}e$,
so that
          \begin{align*}
            &\sum_{y}(\pibar\ind{t}(y)+\pibar\ind{t+1}(y))
            \log^2\prn*{\frac{1}{\pibar\ind{t}(y)}}\\
            &\leq          
              \log^2\prn*{\sum_{y}\pibar\ind{t}(y)\frac{1}{\pibar\ind{t}(y)}\vee{}e}
              +
              \log^2\prn*{\sum_{y}\pibar\ind{t+1}(y)\frac{1}{\pibar\ind{t}(y)}\vee{}e}\\
                        &\leq          
              \log^2\prn*{e\sum_{y}\frac{\pibar\ind{t}(y)}{\pibar\ind{t}(y)}}
                          +
                          \log^2\prn*{e\sum_{y}\frac{\pibar\ind{t+1}(y)}{\pibar\ind{t}(y)}}\\
            &\leq \log^2(e\abs{\cY}) +
              \log^2(e\Cconc\abs{\cY}\alpha^{-1})
              \leq{} 2e^2 \log^2(\Cconc\abs{\cY}\alpha^{-1})
          \end{align*}


          
        \end{proof}

\begin{comment}
  \subsubsection{Gradient Dominance}
  \newcommand{\ystar}{y^{\star}} Define the entropy for a policy $\pi$
  via
  \[
    H(\pi) = \sum_{y}\pi(y)\log\prn*{\frac{1}{\pi(y)}}.
  \]
  Not that this function is concave with respect to $\pi$.
  \begin{lemma}
    \label{lem:gd_bandit}
  
  \end{lemma}
  \begin{proof}[\pfref{lem:gd_bandit}]
  
  \end{proof}
\end{comment}

  \subsection{General Case}

  \dfc{Plan for basic analysis
  \begin{itemize}
  \item Work out the best version of gradient dominance condition.
  \item Assume above holds uniformly for all $x$ at initialization.
  \item Argue that this is preserved whp at each iteration.
  \item Do the exp weights analysis, accounting for statistical error
    at each update step and generalization analysis (possibly unboundedness?).
  \end{itemize}
  Open problems for better analysis:
  \begin{itemize}
  \item Strong convexity-type approach? seems like linear convergence
    must be possible.
  \end{itemize}
  }

\subsubsection{Gradient Dominance}

\subsubsection{Exponential Weights}



\section{IGW Question}
Consider the contextual bandits
Suppose we solve the following optimization problem:
\[
  \pihat=\argmax_{\pi\in\Pi}\crl*{
    \Eh\brk*{r} - \beta\Dkl{\piref}{\pi}}.
\]
For a fixed choice of $\piref$, this is equivalent to generalized IGW
(``SmoothIGW''), and leads to guarantees that scale with
$\nrm*{\frac{\pistar}{\piref}}$. Suppose instead we do the following
process iteratively:
\[
  \pi\ind{t+1}=\argmax_{\pi\in\Pi}\crl*{
    \Eh\brk*{r} - \beta\Dkl{\pi\ind{t}}{\pi}}.
\]
Can we show that this leads to guarantees that scale with the
coverability parameter for the class $\Pi$?

\section{To-do list}
To-do list:
\begin{itemize}
\item Exponential weights-type analysis:
  \begin{itemize}
  \item Re-do exp-weights analysis on the objective, but:
    \begin{itemize}
    \item Do the contextual case and handle noise coming from
      contexts. Try to do hellinger stuff and get the fast rate? 
    \item Handle the case where we solve the Exp Weights objective
      using ERM (do the case above first).
    \end{itemize}
    \item Applying the result to entropy:
      \begin{itemize}
      \item Try to get a fast rate for minimizing entropy from
        this guarantee.
      \item Try to directly get a descent lemma for minimizing entropy
        (don't need to go through the full OL analysis).
        \begin{itemize}
        \item This gives parameter convergence, which may be useful
          for other things we want to do.
        \end{itemize}
    \end{itemize}
  \end{itemize}
\item Showing convergence to argmax:
  \begin{itemize}
  \item Focus on the case where $\Pi$ is parameterized via softmax
    policies to start.
  \item Is there a way to directly derive this from the exp
    
    weights/regret-type reasoning above?
  \item Possible idea: Try to argue that we're converging to argmax
    policy in KL/TV distance (via descent lemma-type reasoning?)
  \item Focus on CB case with general $f$ class---vanilla bandit is
    too easy, but this is rich enough to be interesting.
    \begin{itemize}
    \item Idea: Argue that if $\fstar$ has a uniform margin,
      $\fhat\ind{1}$ also has a uniform margin on most $x$s. Then try
      to directly use properties of exp weights updates to argue that
      each update preserves the margin on most $x$s?
    \end{itemize}
  \item Seems like optimization view might be useful here: Argue that
    argmax policy is a local optima, then argue that we converge to it
    in TV? 
  \item Perhaps the easiest way to import the optimization view is to
    handle the CB case, but where $\Pi$ is all possible
    policies. Still have stat errors, but not entirely non-trivial? Or
    maybe it is since we are basically doing the bandit case pointwise
    when $n\gg\log\abs{\Pi}$.
  \end{itemize}
\end{itemize}

\section{More ideas and questions}
\begin{itemize}
\item Can we understand/analyze SPIN from this sharpening perspective?
\end{itemize}

\bibliography{refs} 

\end{document}

%%% Local Variables:
%%% mode: latex
%%% TeX-master: t
%%% End:
